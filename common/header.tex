\makeatletter
%--------------------------------------------------------------------------------

\usepackage[T1]{fontenc}
\usepackage[utf8]{inputenc}
\usepackage[french]{babel}

\usepackage{lmodern}
\usepackage{dsfont}

%\usepackage{qrcode}

\usepackage{eurosym}
\usepackage{diagbox}
\usepackage{enumitem}
%\def\labelitemi{\small \textbullet}
\setlist[itemize,1]{label={\color{gray}\small \textbullet}}
\usepackage{multicol}

\usepackage{graphicx}
\usepackage{float}

\usepackage{array}
\newcolumntype{L}[1]{>{\raggedright\let\newline\\\arraybackslash\hspace{0pt}}m{#1}}
\newcolumntype{C}[1]{>{\centering\let\newline\\\arraybackslash\hspace{0pt}}m{#1}}
\newcolumntype{R}[1]{>{\raggedleft\let\newline\\\arraybackslash\hspace{0pt}}m{#1}}
%----------
%  Version
%-----------

\usepackage{fancyhdr} 
\usepackage{stmaryrd}

%--------
%  Tkz  
%--------
\usepackage{blkarray}
\usepackage[babel=true,kerning=true]{microtype}
\usepackage[caption=false]{subfig}
\usepackage{xcolor,colortbl}
\definecolor{dgreen}{RGB}{0,100,0}
\definecolor{linkcol}{RGB}{0,118,155}
\definecolor{astral}{RGB}{46,116,181}

\usepackage{diagbox,calc,soul,graphicx}

\usepackage{tikz}
\usetikzlibrary{3d,calc,fadings,decorations.pathreplacing,matrix,arrows,decorations.text}
\usetikzlibrary{patterns}
\usetikzlibrary{positioning}
\usetikzlibrary{babel}
\usetikzlibrary{shapes}
\usetikzlibrary{shadings}
\usetikzlibrary{cd}
\usepackage{tikz-3dplot}
\usepackage{pgfplots}
\usepgfplotslibrary{fillbetween}
\pgfplotsset{compat=newest}
\usepgfplotslibrary{external} 
\tikzexternalize[prefix=output_latex/]
\usepgfplotslibrary{fillbetween}
%\graphicspath{{./output-latex/}}


%\usepackage{etex}
%\reserveinserts{28}
%\usepackage{pstricks}
%\usepackage{pst-solides3d}

\usepackage{gnuplot-lua-tikz}

\newcommand\chideux[1]{#1<=0||(#1!=int(#1))?1/0:x<=0?0.0:exp((0.5*#1-1.0)*log(x)-0.5*x-lgamma(0.5*#1)-#1*0.5*log(2))}
\newcommand\gauss[2]{1/(#2*sqrt(2*pi))*exp(-((x-#1)^2)/(2*#2^2))} 
\newcommand\student[1]{gamma(.5*(#1+1))/(sqrt(#1*pi)*gamma(.5*#1))*((1+x^2/#1)^(-.5*(#1+1)))}

% display dices
\usepackage{xparse}\usetikzlibrary{shapes}
\NewDocumentCommand{\drawdie}{O{}m}{%
\begin{tikzpicture}[x=1em,y=1em,radius=0.1,#1,baseline=0.575ex]
		\draw[rounded corners=0.5] (0,0) rectangle (1,1);
	  \ifodd#2
      \fill[] (0.5,0.5) circle;
        \fi
  \ifnum#2>1
      \fill[] (0.2,0.2) circle;
          \fill[] (0.8,0.8) circle;
     \ifnum#2>3
          \fill[] (0.2,0.8) circle;
       \fill[] (0.8,0.2) circle;
           \ifnum#2>5
         \fill[] (0.8,0.5) circle;
       \fill[] (0.2,0.5) circle;
            \ifnum#2>7
           \fill[] (0.5,0.8) circle;
          \fill[] (0.5,0.2) circle;
        \fi
    \fi
      \fi
      \fi
\end{tikzpicture}%
} 
\NewDocumentCommand{\tdrawdie}{O{}m}{%
    \begin{tikzpicture}[x=.6em,y=.6em,radius=0.1,#1,baseline=0.575ex]
		\draw[rounded corners=0.5] (0,0) rectangle (1,1);
	  \ifodd#2
      \fill[] (0.5,0.5) circle;
        \fi
  \ifnum#2>1
      \fill[] (0.2,0.2) circle;
          \fill[] (0.8,0.8) circle;
     \ifnum#2>3
          \fill[] (0.2,0.8) circle;
       \fill[] (0.8,0.2) circle;
           \ifnum#2>5
         \fill[] (0.8,0.5) circle;
       \fill[] (0.2,0.5) circle;
            \ifnum#2>7
           \fill[] (0.5,0.8) circle;
          \fill[] (0.5,0.2) circle;
        \fi
    \fi
      \fi
      \fi
\end{tikzpicture}%
}

\usetikzlibrary{lindenmayersystems}
\pgfdeclarelindenmayersystem{cantor set}{
  \rule{F -> FfF}
  \rule{f -> fff}
}

%------------------
% Math environment
%------------------

\usepackage{latexsym}
\usepackage{amsmath}
\usepackage{amsbsy}
\usepackage{amsfonts}
\usepackage{amssymb}
\usepackage{nicefrac}
\usepackage{amscd}
\usepackage{amsthm}
\usepackage{mathtools}

\newtheoremstyle{definitionSs}{\topsep}{\topsep}%
     {}%         Body font
     {}%         Indent amount (empty = no indent, \parindent = para indent)
     {\sffamily\bfseries}% Thm head font
     {.}%        Punctuation after thm head
     { }%     Space after thm head (\newline = linebreak)
     {\thmname{#1}\thmnumber{~#2}\thmnote{~#3}}%         Thm head spec

\newtheoremstyle{plainSs}{\topsep}{\topsep}%
     {\itshape}%         Body font
     {}%         Indent amount (empty = no indent, \parindent = para indent)
     {\sffamily\bfseries}% Thm head font
     {.}%        Punctuation after thm head
     { }%     Space after thm head (\newline = linebreak)
     {\thmname{#1}\thmnumber{~#2}\thmnote{~#3}}%         Thm head spec

\theoremstyle{definitionSs}
\newtheorem{remark}{Remarque}[section]
\newtheorem*{remark*}{Remarque}
\newtheorem{example}{Exemple}[section]


%\usepackage[framemethod=tikz]{mdframed}
\usepackage[]{mdframed}

\newmdtheoremenv[
hidealllines=true,
leftline=true,
skipabove=0pt,
innertopmargin=-5pt,
innerbottommargin=2pt,
linewidth=4pt,
linecolor=gray!90,
innerrightmargin=0pt,
]{definition}{Définition}[section]

\newmdtheoremenv[
hidealllines=true,
leftline=true,
skipabove=0pt,
innertopmargin=-5pt,
innerbottommargin=2pt,
linewidth=4pt,
linecolor=gray!40,
innerrightmargin=0pt,
]{lemme}{Lemme}[section]

\newmdtheoremenv[
hidealllines=true,
leftline=true,
skipabove=0pt,
innertopmargin=-5pt,
innerbottommargin=2pt,
linewidth=4pt,
linecolor=gray!40,
innerrightmargin=0pt,
]{proposition}{Proposition}[section]

\newmdtheoremenv[
hidealllines=true,
leftline=true,
skipabove=0pt,
innertopmargin=-5pt,
innerbottommargin=2pt,
linewidth=4pt,
linecolor=gray!40,
innerrightmargin=0pt,
]{corollaire}{Corollaire}[section]

\newmdtheoremenv[
hidealllines=true,
leftline=true,
skipabove=0pt,
innertopmargin=-5pt,
innerbottommargin=2pt,
linewidth=4pt,
linecolor=gray!90,
innerrightmargin=0pt,
]{propdef}{Définition - Proposition}[section]

\newmdtheoremenv[
hidealllines=true,
leftline=true,
skipabove=0pt,
innertopmargin=-5pt,
innerbottommargin=2pt,
linewidth=4pt,
linecolor=gray!100,
innerrightmargin=0pt,
]{theorem}{Théorème}[section]

%---------------
% Mise en page
%--------------

\setlength{\parindent}{0pt}

\renewcommand*{\descriptionlabel}[1]{\hspace\labelsep{\sffamily #1}}
\providecommand{\defemph}[1]{{\sffamily\bfseries\color{astral}#1}}
\renewcommand{\emph}[1]{{\sffamily #1}}

%\usepackage{titlesec, blindtext, color}
%\newcommand{\hsp}{\hspace{20pt}}
%\titleformat{\chapter}[hang]{\sffamily\Huge\bfseries}{\thechapter\hsp\textcolor{gray75}{|}\hsp}{0pt}{\Huge\bfseries}
\usepackage{sectsty}
\allsectionsfont{\color{astral}\normalfont\sffamily\bfseries}

\usepackage[subfigure]{tocloft}
\renewcommand{\cfttoctitlefont}{\sffamily\bfseries\Huge\color{astral}}
% \renewcommand{\cftchapfont}{\sffamily\bfseries\color{astral}}
\renewcommand{\cftsecfont}{\sffamily\bfseries}
\renewcommand{\cftsubsecfont}{\sffamily}
\usepackage{hyperref}
\hypersetup{linktocpage, colorlinks=true, linkcolor=gray, urlcolor=linkcol, citecolor=gray}


%----------------
% Some commands
%----------------

\makeatletter
\newcommand\RedeclareMathOperator{%
  \@ifstar{\def\rmo@s{m}\rmo@redeclare}{\def\rmo@s{o}\rmo@redeclare}%
}
% this is taken from \renew@command
\newcommand\rmo@redeclare[2]{%
  \begingroup \escapechar\m@ne\xdef\@gtempa{{\string#1}}\endgroup
  \expandafter\@ifundefined\@gtempa
     {\@latex@error{\noexpand#1undefined}\@ehc}%
     \relax
  \expandafter\rmo@declmathop\rmo@s{#1}{#2}}
% This is just \@declmathop without \@ifdefinable
\newcommand\rmo@declmathop[3]{%
  \DeclareRobustCommand{#2}{\qopname\newmcodes@#1{#3}}%
}
\@onlypreamble\RedeclareMathOperator
\makeatother

\newcommand{\skipline}{\vspace{\baselineskip}}
\newcommand{\noi}{\noindent}


%----- Raccourci proba ------------------
\newcommand{\Om}{\Omega}
\newcommand{\Tribu}{\cF}
\newcommand{\Tribubis}{\cG}
\newcommand{\Partie}{\cE}
\newcommand{\Proba}{\bbP}
\newcommand{\Espe}{\bbE}
\newcommand{\OF}{(\Om,\Tribu)}
\newcommand{\RBR}{\big(\bbR,\cB(\bbR)\big)}
\newcommand{\OFP}{(\Om,\Tribu,\Proba)}
\newcommand{\POm}{\cP(\Om)}
\newcommand{\Leb}{\lambda}
%----- Lettres -------------------------
\newcommand{\bbC}{\mathbb{C}}
\newcommand{\bbE}{\mathbb{E}}
\newcommand{\bbF}{\mathbb{F}}
\newcommand{\bbG}{\mathbb{G}}
\newcommand{\bbK}{\mathbb{K}}
\newcommand{\bbL}{\mathbb{L}}
\newcommand{\bbN}{\mathbb{N}}
\newcommand{\bbP}{\mathbb{P}}
\newcommand{\bbQ}{\mathbb{Q}}
\newcommand{\bbR}{\mathbb{R}}
\newcommand{\bbS}{\mathbb{S}}
\newcommand{\bbT}{\mathbb{T}}
\newcommand{\bbU}{\mathbb{U}}
\newcommand{\bbZ}{\mathbb{Z}}

\newcommand{\bfC}{\mathbf{C}}
\newcommand{\bfE}{\mathbf{E}}
\newcommand{\bfL}{\mathbf{L}}
\newcommand{\bfN}{\mathbf{N}}
\newcommand{\bfP}{\mathbf{P}}
\newcommand{\bfQ}{\mathbf{Q}}
\newcommand{\bfR}{\mathbf{R}}
\newcommand{\bfZ}{\mathbf{Z}}

\newcommand{\calA}{\mathcal{A}}
\newcommand{\calB}{\mathcal{B}}
\newcommand{\calC}{\mathcal{C}}
\newcommand{\calE}{\mathcal{E}}
\newcommand{\calF}{\mathcal{F}}
\newcommand{\calG}{\mathcal{G}}
\newcommand{\calJ}{\mathcal{J}}
\newcommand{\calL}{\mathcal{L}}
\newcommand{\calM}{\mathcal{M}}
\newcommand{\calN}{\mathcal{N}}
\newcommand{\calO}{\mathcal{O}}
\newcommand{\calP}{\mathcal{P}}
\newcommand{\calS}{\mathcal{S}}
\newcommand{\calT}{\mathcal{T}}
\newcommand{\calV}{\mathcal{V}}
\newcommand{\calX}{\mathcal{X}}

\newcommand{\fA}{\mathfrak{A}}
\newcommand{\fB}{\mathfrak{B}}
\newcommand{\fF}{\mathfrak{F}}

% la droite réelle achevée
\newcommand{\barR}{\overline{\mathbb{R}}}

%
%------------------------------- Majuscules calligraphiques
\renewcommand{\geq}{\geqslant}
\renewcommand{\ge}{\geqslant}
\renewcommand{\leq}{\leqslant}
\renewcommand{\le}{\leqslant}
%
\usepackage{mathrsfs}
\newcommand{\scrL}{\mathscr{L}}
\newcommand{\scrF}{\mathscr{F}}
\newcommand{\scrM}{\mathscr{M}}
\newcommand{\scrN}{\mathscr{N}}
\newcommand{\scrD}{\mathscr{D}}
\newcommand{\scrE}{\mathscr{E}}
\newcommand{\scrA}{\mathscr{A}}
\newcommand{\scrB}{\mathscr{B}}
\newcommand{\scrC}{\mathscr{C}}
\newcommand{\scrI}{\mathscr{I}}
\newcommand{\scrJ}{\mathscr{J}}
\newcommand{\scrK}{\mathscr{K}}
\newcommand{\scrS}{\mathscr{S}}
\newcommand{\scrO}{\mathscr{O}}
\newcommand{\scrP}{\mathscr{P}}
\newcommand{\scrR}{\mathscr{R}}
\newcommand{\scrT}{\mathscr{T}}
\newcommand{\scrZ}{\mathscr{Z}}
%
\newcommand{\vois}{\mathscr{V}}
\newcommand{\ouv}{\mathscr{O}}
\newcommand{\ferm}{\mathscr{F}}

%
%------------------------------- Intervalles
%
\newcommand{\oo}[1]{\mathopen{]}#1\mathclose{[}}
\newcommand{\fo}[1]{\mathopen{[}#1\mathclose{[}}
\newcommand{\of}[1]{\mathopen{]}#1\mathclose{]}}
\newcommand{\ff}[1]{\mathopen{[}#1\mathclose{]}}
\newcommand{\lbb}{[\![}
\newcommand{\rbb}{]\!]}

%
\newcommand{\ex}{e}
\newcommand{\ind}{\mathrm{Ind}}
\newcommand{\spt}{\mathrm{Spt}}
\newcommand{\pd}{\frac{\pi}{2}}
\newcommand{\E}{\mathrm{E}}
\newcommand{\diff}{\mathrm{Diff}}
\newcommand{\diam}{\mathrm{diam}}
\newcommand{\comp}[1]{#1^\mathrm{c}}
\newcommand{\vvv}{|\!|\!|}
\newcommand{\dd}[2]{\frac{\partial #1}{\partial #2}}
%
%----- chapeaux, tildes -------------------------
\newcommand{\wb}[1]{\overline{#1}}
\newcommand{\wh}[1]{\widehat{#1}}
\newcommand{\wt}[1]{\widetilde{#1}}

%----- ccouleurs -------------------------
\newcommand{\rose}[1]{\textcolor{magenta}{#1}}

%----- Maths -----------------------
\newcommand {\expl}[2]{\mathrm{e}^{-\Lambda(#1,#2)}}
\newcommand {\expla}[2]{\mathrm{e}^{-\alpha #2-\Lambda(#1,#2)}}
\newcommand {\Li}[1]{[ #1 ]}
%\newcommand{\1}{\mathbbm{1}}
\newcommand {\tstar}[1]{t^{*}(#1)}
\DeclareMathOperator{\aire}{Aire}
\providecommand{\gf}{g\circ f}
\providecommand{\R}{\ensuremath \mathbb{R}}
\providecommand{\C}{\ensuremath \mathbb{C}}
\providecommand{\reg}[1]{\mathcal{C}^{#1}}
\providecommand{\N}{\mathbb{N}}
\providecommand{\M}{\mathcal{M}}
\providecommand{\Q}{\mathbb{Q}}
\renewcommand{\L}{\mathcal{L}}
\providecommand{\D}{\mathcal{D}}
\providecommand{\Cc}{\mathcal{C}}
\providecommand{\F}{\mathcal{F}}
\providecommand{\Ee}{\mathcal{E}}
\providecommand{\G}{\mathcal{G}}
\providecommand{\Z}{\mathbb{Z}}
\providecommand{\x}{\ensuremath\boldsymbol{x}}
\providecommand{\y}{\ensuremath\boldsymbol{y}}
\providecommand{\1}{\mathds{1}}
\providecommand{\p}{\partial}
\providecommand{\Pp}{\mathcal{P}}
\providecommand{\P}{\mathbb{P}}
\providecommand{\E}{\mathbb{E}}
\providecommand{\U}{\mathcal{U}}
\providecommand{\V}{\mathcal{V}}
\providecommand{\ie}{\textit{i.e. }}
\renewcommand{\P}{\mathbb{P}}
\renewcommand{\S}{\mathcal{S}}
\providecommand{\E}{\mathbb{E}}
\providecommand{\one}{\mathds{1}}
\DeclareMathOperator{\card}{Card}
\DeclareMathOperator{\vol}{Vol}
\DeclareMathOperator{\var}{Var}
\DeclareMathOperator{\vect}{\mathsf{Vect}}
\DeclareMathOperator{\med}{median}
\DeclareMathOperator{\hess}{Hess}
\DeclareMathOperator{\jac}{Jac}
\DeclareMathOperator{\cov}{cov}
\DeclareMathOperator{\diag}{\mathsf{Diag}}
\DeclareMathOperator{\im}{\mathsf{Im}}
\DeclareRobustCommand{\re}{\mathsf{Re}}
\RedeclareMathOperator{\ker}{\mathsf{Ker}}
\RedeclareMathOperator{\det}{\mathsf{det}}
\DeclareMathOperator{\rank}{\mathsf{rank}}
\DeclareMathOperator{\tr}{\mathsf{tr}}
\DeclareMathOperator{\id}{\mathsf{Id}}
\DeclareMathOperator{\can}{\mathsf{can}}
\DeclareMathOperator{\com}{\mathsf{com}}
\DeclareMathOperator*{\argmax}{Argmax}
\providecommand{\B}{\mathsf{B}}

\providecommand{\ncd}{\norm{\cdot}}
\providecommand{\norm}[1]{\left\lVert#1\right\rVert}
\providecommand{\bnorm}[1]{\bigg\lVert#1\bigg\rVert}
\providecommand{\snorm}[1]{\lVert#1\rVert}

\newcommand{\tnorm}[1]{{\left\vert\kern-0.25ex\left\vert\kern-0.25ex\left\vert #1 
    \right\vert\kern-0.25ex\right\vert\kern-0.25ex\right\vert}}

\providecommand{\abs}[1]{\left\lvert#1\right\rvert}
\providecommand{\sabs}[1]{\lvert#1\rvert}
\providecommand{\Babs}[1]{\bigg\lvert#1\bigg\rvert}
\providecommand{\babs}[1]{\bigg\lvert#1\bigg\rvert}

\providecommand{\prscd}{\prs{\cdot,\cdot}}
\providecommand{\prs}[1]{\left\langle #1\right\rangle}
\providecommand{\sprs}[1]{\langle #1\rangle}
\providecommand{\bprs}[1]{\bigg\langle #1\bigg\rangle}

\providecommand{\rev}{$\R$ espace vectoriel}

\providecommand{\dpar}[2]{\frac{\partial #1}{\partial #2}}

\newcommand\rst[2]{{#1}_{\restriction_{#2}}}


% Multiversioning 


\usepackage{ifthen}

\newcommand{\pl}[2]{%
	\ifthenelse{\equal{\version}{poly}}{#1}{}%
	\ifthenelse{\equal{\version}{print}}{#2}{}%
}
\newcommand{\plprof}[1]{\ifthenelse{\equal{\version}{print}}{#1}{}}
\newcommand{\sld}[1]{\ifthenelse{\equal{\version}{slide}}{#1}{}}

\newcommand{\eno}[1]{%
	\ifthenelse{\equal{\version}{eno}}{#1}{}%
}
\newcommand{\cor}[1]{%
        \ifthenelse{\equal{\version}{cor}}{
\medskip 

{\small \color{gray} #1}

\medskip 
}{}
}

\mdfdefinestyle{response}{
	leftmargin=.01\textwidth,
	rightmargin=.01\textwidth,
	linewidth=1pt
	hidealllines=false,
	leftline=true,
	rightline=true,topline=true,bottomline=true,
        skipabove=\baselineskip,%0pt,
	%innertopmargin=-5pt,
	%innerbottommargin=2pt,
	linecolor=gray,
	innerrightmargin=0pt,
	}
\providecommand{\rep}[1]{$ $ \newline \begin{mdframed}[style=response] \vspace*{#1} \end{mdframed}}

% generate breakable white space allowing students to write notes.
\newcommand*{\DivideLengths}[2]{%
  \strip@pt\dimexpr\number\numexpr\number\dimexpr#1\relax*65536/\number\dimexpr#2\relax\relax sp\relax
}

\usepackage{xparse}
\ExplSyntaxOn
\NewDocumentCommand\manyvspace { m }
 {
  \par
  \int_step_inline:nn{#1}{\phantom{.}\vspace*{1em}\goodbreak}
 }
\ExplSyntaxOff 

%\providecommand{\rep}[1]{$ $

    %\begin{mdframed}[style=response]  
	%\vspace*{\DivideLengths{#1}{3cm}cm}
	%\pagebreak[1]	
	%\vspace*{\DivideLengths{#1}{3cm}cm}
	%\pagebreak[1]		
	%\vspace*{\DivideLengths{#1}{3cm}cm}
    %\end{mdframed}%
%}
\providecommand{\repcom}[1]{\begin{mdframed}[style=response] #1 \end{mdframed}}
\providecommand{\comexo}[1]{\bigskip {\color{gray}#1}}
\providecommand{\blanc}[1]{\vspace*{#1}}
\providecommand{\myeq}[1]{\mathrel{\overset{\makebox[0pt]{\mbox{\normalfont\tiny #1}}}{=}}}
\def\redspace{\sld{\setlength{\belowdisplayskip}{0pt} \setlength{\belowdisplayshortskip}{0pt}\setlength{\abovedisplayskip}{0pt}\setlength{\abovedisplayshortskip}{0pt}}}
%------------------------------------------------------------------------------
%\DeclareUnicodeCharacter{00A0}{~}

\pdfstringdefDisableCommands{%
    %\renewcommand*{\bm}[1]{#1}%
    \renewcommand*{\R}{R}%
    % any other necessary redefinitions 
}
\makeatother


