\documentclass[11pt]{td_um}

%------------------------------
\def\version{cor}
%\def\version{eno}
%------------------------------

\makeatletter
%--------------------------------------------------------------------------------

\usepackage[T1]{fontenc}
\usepackage[utf8]{inputenc}
\usepackage[french]{babel}

\usepackage{lmodern}
\usepackage{dsfont}

%\usepackage{qrcode}

\usepackage{eurosym}
\usepackage{diagbox}
\usepackage{enumitem}
%\def\labelitemi{\small \textbullet}
\setlist[itemize,1]{label={\color{gray}\small \textbullet}}
\usepackage{multicol}

\usepackage{graphicx}
\usepackage{float}

\usepackage{array}
\newcolumntype{L}[1]{>{\raggedright\let\newline\\\arraybackslash\hspace{0pt}}m{#1}}
\newcolumntype{C}[1]{>{\centering\let\newline\\\arraybackslash\hspace{0pt}}m{#1}}
\newcolumntype{R}[1]{>{\raggedleft\let\newline\\\arraybackslash\hspace{0pt}}m{#1}}
%----------
%  Version
%-----------

\usepackage{fancyhdr} 
\usepackage{stmaryrd}

%--------
%  Tkz  
%--------
\usepackage{blkarray}
\usepackage[babel=true,kerning=true]{microtype}
\usepackage[caption=false]{subfig}
\usepackage{xcolor,colortbl}
\definecolor{dgreen}{RGB}{0,100,0}
\definecolor{linkcol}{RGB}{0,118,155}
\definecolor{astral}{RGB}{46,116,181}

\usepackage{diagbox,calc,soul,graphicx}

\usepackage{tikz}
\usetikzlibrary{3d,calc,fadings,decorations.pathreplacing,matrix,arrows,decorations.text}
\usetikzlibrary{patterns}
\usetikzlibrary{positioning}
\usetikzlibrary{babel}
\usetikzlibrary{shapes}
\usetikzlibrary{shadings}
\usetikzlibrary{cd}
\usepackage{tikz-3dplot}
\usepackage{pgfplots}
\usepgfplotslibrary{fillbetween}
\pgfplotsset{compat=newest}
\usepgfplotslibrary{external} 
\tikzexternalize[prefix=output_latex/]
\usepgfplotslibrary{fillbetween}
%\graphicspath{{./output-latex/}}


%\usepackage{etex}
%\reserveinserts{28}
%\usepackage{pstricks}
%\usepackage{pst-solides3d}

\usepackage{gnuplot-lua-tikz}

\newcommand\chideux[1]{#1<=0||(#1!=int(#1))?1/0:x<=0?0.0:exp((0.5*#1-1.0)*log(x)-0.5*x-lgamma(0.5*#1)-#1*0.5*log(2))}
\newcommand\gauss[2]{1/(#2*sqrt(2*pi))*exp(-((x-#1)^2)/(2*#2^2))} 
\newcommand\student[1]{gamma(.5*(#1+1))/(sqrt(#1*pi)*gamma(.5*#1))*((1+x^2/#1)^(-.5*(#1+1)))}

% display dices
\usepackage{xparse}\usetikzlibrary{shapes}
\NewDocumentCommand{\drawdie}{O{}m}{%
\begin{tikzpicture}[x=1em,y=1em,radius=0.1,#1,baseline=0.575ex]
		\draw[rounded corners=0.5] (0,0) rectangle (1,1);
	  \ifodd#2
      \fill[] (0.5,0.5) circle;
        \fi
  \ifnum#2>1
      \fill[] (0.2,0.2) circle;
          \fill[] (0.8,0.8) circle;
     \ifnum#2>3
          \fill[] (0.2,0.8) circle;
       \fill[] (0.8,0.2) circle;
           \ifnum#2>5
         \fill[] (0.8,0.5) circle;
       \fill[] (0.2,0.5) circle;
            \ifnum#2>7
           \fill[] (0.5,0.8) circle;
          \fill[] (0.5,0.2) circle;
        \fi
    \fi
      \fi
      \fi
\end{tikzpicture}%
} 
\NewDocumentCommand{\tdrawdie}{O{}m}{%
    \begin{tikzpicture}[x=.6em,y=.6em,radius=0.1,#1,baseline=0.575ex]
		\draw[rounded corners=0.5] (0,0) rectangle (1,1);
	  \ifodd#2
      \fill[] (0.5,0.5) circle;
        \fi
  \ifnum#2>1
      \fill[] (0.2,0.2) circle;
          \fill[] (0.8,0.8) circle;
     \ifnum#2>3
          \fill[] (0.2,0.8) circle;
       \fill[] (0.8,0.2) circle;
           \ifnum#2>5
         \fill[] (0.8,0.5) circle;
       \fill[] (0.2,0.5) circle;
            \ifnum#2>7
           \fill[] (0.5,0.8) circle;
          \fill[] (0.5,0.2) circle;
        \fi
    \fi
      \fi
      \fi
\end{tikzpicture}%
}

\usetikzlibrary{lindenmayersystems}
\pgfdeclarelindenmayersystem{cantor set}{
  \rule{F -> FfF}
  \rule{f -> fff}
}

%------------------
% Math environment
%------------------

\usepackage{latexsym}
\usepackage{amsmath}
\usepackage{amsbsy}
\usepackage{amsfonts}
\usepackage{amssymb}
\usepackage{nicefrac}
\usepackage{amscd}
\usepackage{amsthm}
\usepackage{mathtools}

\newtheoremstyle{definitionSs}{\topsep}{\topsep}%
     {}%         Body font
     {}%         Indent amount (empty = no indent, \parindent = para indent)
     {\sffamily\bfseries}% Thm head font
     {.}%        Punctuation after thm head
     { }%     Space after thm head (\newline = linebreak)
     {\thmname{#1}\thmnumber{~#2}\thmnote{~#3}}%         Thm head spec

\newtheoremstyle{plainSs}{\topsep}{\topsep}%
     {\itshape}%         Body font
     {}%         Indent amount (empty = no indent, \parindent = para indent)
     {\sffamily\bfseries}% Thm head font
     {.}%        Punctuation after thm head
     { }%     Space after thm head (\newline = linebreak)
     {\thmname{#1}\thmnumber{~#2}\thmnote{~#3}}%         Thm head spec

\theoremstyle{definitionSs}
\newtheorem{remark}{Remarque}[section]
\newtheorem*{remark*}{Remarque}
\newtheorem{example}{Exemple}[section]


%\usepackage[framemethod=tikz]{mdframed}
\usepackage[]{mdframed}

\newmdtheoremenv[
hidealllines=true,
leftline=true,
skipabove=0pt,
innertopmargin=-5pt,
innerbottommargin=2pt,
linewidth=4pt,
linecolor=gray!90,
innerrightmargin=0pt,
]{definition}{Définition}[section]

\newmdtheoremenv[
hidealllines=true,
leftline=true,
skipabove=0pt,
innertopmargin=-5pt,
innerbottommargin=2pt,
linewidth=4pt,
linecolor=gray!40,
innerrightmargin=0pt,
]{lemme}{Lemme}[section]

\newmdtheoremenv[
hidealllines=true,
leftline=true,
skipabove=0pt,
innertopmargin=-5pt,
innerbottommargin=2pt,
linewidth=4pt,
linecolor=gray!40,
innerrightmargin=0pt,
]{proposition}{Proposition}[section]

\newmdtheoremenv[
hidealllines=true,
leftline=true,
skipabove=0pt,
innertopmargin=-5pt,
innerbottommargin=2pt,
linewidth=4pt,
linecolor=gray!40,
innerrightmargin=0pt,
]{corollaire}{Corollaire}[section]

\newmdtheoremenv[
hidealllines=true,
leftline=true,
skipabove=0pt,
innertopmargin=-5pt,
innerbottommargin=2pt,
linewidth=4pt,
linecolor=gray!90,
innerrightmargin=0pt,
]{propdef}{Définition - Proposition}[section]

\newmdtheoremenv[
hidealllines=true,
leftline=true,
skipabove=0pt,
innertopmargin=-5pt,
innerbottommargin=2pt,
linewidth=4pt,
linecolor=gray!100,
innerrightmargin=0pt,
]{theorem}{Théorème}[section]

%---------------
% Mise en page
%--------------

\setlength{\parindent}{0pt}

\renewcommand*{\descriptionlabel}[1]{\hspace\labelsep{\sffamily #1}}
\providecommand{\defemph}[1]{{\sffamily\bfseries\color{astral}#1}}
\renewcommand{\emph}[1]{{\sffamily #1}}

%\usepackage{titlesec, blindtext, color}
%\newcommand{\hsp}{\hspace{20pt}}
%\titleformat{\chapter}[hang]{\sffamily\Huge\bfseries}{\thechapter\hsp\textcolor{gray75}{|}\hsp}{0pt}{\Huge\bfseries}
\usepackage{sectsty}
\allsectionsfont{\color{astral}\normalfont\sffamily\bfseries}

\usepackage[subfigure]{tocloft}
\renewcommand{\cfttoctitlefont}{\sffamily\bfseries\Huge\color{astral}}
\renewcommand{\cftchapfont}{\sffamily\bfseries\color{astral}}
\renewcommand{\cftsecfont}{\sffamily\bfseries}
\renewcommand{\cftsubsecfont}{\sffamily}
\usepackage{hyperref}
\hypersetup{linktocpage, colorlinks=true, linkcolor=gray, urlcolor=linkcol, citecolor=gray}


%----------------
% Some commands
%----------------

\makeatletter
\newcommand\RedeclareMathOperator{%
  \@ifstar{\def\rmo@s{m}\rmo@redeclare}{\def\rmo@s{o}\rmo@redeclare}%
}
% this is taken from \renew@command
\newcommand\rmo@redeclare[2]{%
  \begingroup \escapechar\m@ne\xdef\@gtempa{{\string#1}}\endgroup
  \expandafter\@ifundefined\@gtempa
     {\@latex@error{\noexpand#1undefined}\@ehc}%
     \relax
  \expandafter\rmo@declmathop\rmo@s{#1}{#2}}
% This is just \@declmathop without \@ifdefinable
\newcommand\rmo@declmathop[3]{%
  \DeclareRobustCommand{#2}{\qopname\newmcodes@#1{#3}}%
}
\@onlypreamble\RedeclareMathOperator
\makeatother

\newcommand{\skipline}{\vspace{\baselineskip}}
\newcommand{\noi}{\noindent}


%----- Raccourci proba ------------------
\newcommand{\Om}{\Omega}
\newcommand{\Tribu}{\cF}
\newcommand{\Tribubis}{\cG}
\newcommand{\Partie}{\cE}
\newcommand{\Proba}{\bbP}
\newcommand{\Espe}{\bbE}
\newcommand{\OF}{(\Om,\Tribu)}
\newcommand{\RBR}{\big(\bbR,\cB(\bbR)\big)}
\newcommand{\OFP}{(\Om,\Tribu,\Proba)}
\newcommand{\POm}{\cP(\Om)}
\newcommand{\Leb}{\lambda}
%----- Lettres -------------------------
\newcommand{\bbC}{\mathbb{C}}
\newcommand{\bbE}{\mathbb{E}}
\newcommand{\bbF}{\mathbb{F}}
\newcommand{\bbG}{\mathbb{G}}
\newcommand{\bbK}{\mathbb{K}}
\newcommand{\bbL}{\mathbb{L}}
\newcommand{\bbN}{\mathbb{N}}
\newcommand{\bbP}{\mathbb{P}}
\newcommand{\bbQ}{\mathbb{Q}}
\newcommand{\bbR}{\mathbb{R}}
\newcommand{\bbS}{\mathbb{S}}
\newcommand{\bbT}{\mathbb{T}}
\newcommand{\bbU}{\mathbb{U}}
\newcommand{\bbZ}{\mathbb{Z}}

\newcommand{\bfC}{\mathbf{C}}
\newcommand{\bfE}{\mathbf{E}}
\newcommand{\bfL}{\mathbf{L}}
\newcommand{\bfN}{\mathbf{N}}
\newcommand{\bfP}{\mathbf{P}}
\newcommand{\bfQ}{\mathbf{Q}}
\newcommand{\bfR}{\mathbf{R}}
\newcommand{\bfZ}{\mathbf{Z}}

\newcommand{\calA}{\mathcal{A}}
\newcommand{\calB}{\mathcal{B}}
\newcommand{\calC}{\mathcal{C}}
\newcommand{\calE}{\mathcal{E}}
\newcommand{\calF}{\mathcal{F}}
\newcommand{\calG}{\mathcal{G}}
\newcommand{\calJ}{\mathcal{J}}
\newcommand{\calL}{\mathcal{L}}
\newcommand{\calM}{\mathcal{M}}
\newcommand{\calN}{\mathcal{N}}
\newcommand{\calO}{\mathcal{O}}
\newcommand{\calP}{\mathcal{P}}
\newcommand{\calS}{\mathcal{S}}
\newcommand{\calT}{\mathcal{T}}
\newcommand{\calV}{\mathcal{V}}
\newcommand{\calX}{\mathcal{X}}

\newcommand{\fA}{\mathfrak{A}}
\newcommand{\fB}{\mathfrak{B}}
\newcommand{\fF}{\mathfrak{F}}

% la droite réelle achevée
\newcommand{\barR}{\overline{\mathbb{R}}}

%
%------------------------------- Majuscules calligraphiques
\renewcommand{\geq}{\geqslant}
\renewcommand{\ge}{\geqslant}
\renewcommand{\leq}{\leqslant}
\renewcommand{\le}{\leqslant}
%
\usepackage{mathrsfs}
\newcommand{\scrL}{\mathscr{L}}
\newcommand{\scrF}{\mathscr{F}}
\newcommand{\scrM}{\mathscr{M}}
\newcommand{\scrD}{\mathscr{D}}
\newcommand{\scrE}{\mathscr{E}}
\newcommand{\scrA}{\mathscr{A}}
\newcommand{\scrB}{\mathscr{B}}
\newcommand{\scrC}{\mathscr{C}}
\newcommand{\scrI}{\mathscr{I}}
\newcommand{\scrJ}{\mathscr{J}}
\newcommand{\scrK}{\mathscr{K}}
\newcommand{\scrS}{\mathscr{S}}
\newcommand{\scrO}{\mathscr{O}}
\newcommand{\scrP}{\mathscr{P}}
\newcommand{\scrR}{\mathscr{R}}
\newcommand{\scrT}{\mathscr{T}}
\newcommand{\scrZ}{\mathscr{Z}}
%
\newcommand{\vois}{\mathscr{V}}
\newcommand{\ouv}{\mathscr{O}}
\newcommand{\ferm}{\mathscr{F}}

%
%------------------------------- Intervalles
%
\newcommand{\oo}[1]{\mathopen{]}#1\mathclose{[}}
\newcommand{\fo}[1]{\mathopen{[}#1\mathclose{[}}
\newcommand{\of}[1]{\mathopen{]}#1\mathclose{]}}
\newcommand{\ff}[1]{\mathopen{[}#1\mathclose{]}}
\newcommand{\lbb}{[\![}
\newcommand{\rbb}{]\!]}

%
\newcommand{\ex}{e}
\newcommand{\ind}{\mathrm{Ind}}
\newcommand{\spt}{\mathrm{Spt}}
\newcommand{\pd}{\frac{\pi}{2}}
\newcommand{\E}{\mathrm{E}}
\newcommand{\diff}{\mathrm{Diff}}
\newcommand{\diam}{\mathrm{diam}}
\newcommand{\comp}[1]{#1^\mathrm{c}}
\newcommand{\vvv}{|\!|\!|}
\newcommand{\dd}[2]{\frac{\partial #1}{\partial #2}}
%
%----- chapeaux, tildes -------------------------
\newcommand{\wb}[1]{\overline{#1}}
\newcommand{\wh}[1]{\widehat{#1}}
\newcommand{\wt}[1]{\widetilde{#1}}

%----- ccouleurs -------------------------
\newcommand{\rose}[1]{\textcolor{magenta}{#1}}

%----- Maths -----------------------
\newcommand {\expl}[2]{\mathrm{e}^{-\Lambda(#1,#2)}}
\newcommand {\expla}[2]{\mathrm{e}^{-\alpha #2-\Lambda(#1,#2)}}
\newcommand {\Li}[1]{[ #1 ]}
%\newcommand{\1}{\mathbbm{1}}
\newcommand {\tstar}[1]{t^{*}(#1)}
\DeclareMathOperator{\aire}{Aire}
\providecommand{\gf}{g\circ f}
\providecommand{\R}{\ensuremath \mathbb{R}}
\providecommand{\C}{\ensuremath \mathbb{C}}
\providecommand{\reg}[1]{\mathcal{C}^{#1}}
\providecommand{\N}{\mathbb{N}}
\providecommand{\M}{\mathcal{M}}
\providecommand{\Q}{\mathbb{Q}}
\renewcommand{\L}{\mathcal{L}}
\providecommand{\D}{\mathcal{D}}
\providecommand{\Cc}{\mathcal{C}}
\providecommand{\F}{\mathcal{F}}
\providecommand{\Ee}{\mathcal{E}}
\providecommand{\G}{\mathcal{G}}
\providecommand{\Z}{\mathbb{Z}}
\providecommand{\x}{\ensuremath\boldsymbol{x}}
\providecommand{\y}{\ensuremath\boldsymbol{y}}
\providecommand{\1}{\mathds{1}}
\providecommand{\p}{\partial}
\providecommand{\Pp}{\mathcal{P}}
\providecommand{\P}{\mathbb{P}}
\providecommand{\E}{\mathbb{E}}
\providecommand{\U}{\mathcal{U}}
\providecommand{\V}{\mathcal{V}}
\providecommand{\ie}{\textit{i.e. }}
\renewcommand{\P}{\mathbb{P}}
\renewcommand{\S}{\mathcal{S}}
\providecommand{\E}{\mathbb{E}}
\providecommand{\one}{\mathds{1}}
\DeclareMathOperator{\card}{Card}
\DeclareMathOperator{\vol}{Vol}
\DeclareMathOperator{\var}{Var}
\DeclareMathOperator{\vect}{\mathsf{Vect}}
\DeclareMathOperator{\med}{median}
\DeclareMathOperator{\hess}{Hess}
\DeclareMathOperator{\jac}{Jac}
\DeclareMathOperator{\cov}{cov}
\DeclareMathOperator{\diag}{\mathsf{Diag}}
\DeclareMathOperator{\im}{\mathsf{Im}}
\DeclareRobustCommand{\re}{\mathsf{Re}}
\RedeclareMathOperator{\ker}{\mathsf{Ker}}
\RedeclareMathOperator{\det}{\mathsf{det}}
\DeclareMathOperator{\rank}{\mathsf{rank}}
\DeclareMathOperator{\tr}{\mathsf{tr}}
\DeclareMathOperator{\id}{\mathsf{Id}}
\DeclareMathOperator{\can}{\mathsf{can}}
\DeclareMathOperator{\com}{\mathsf{com}}
\DeclareMathOperator*{\argmax}{Argmax}
\providecommand{\B}{\mathsf{B}}

\providecommand{\ncd}{\norm{\cdot}}
\providecommand{\norm}[1]{\left\lVert#1\right\rVert}
\providecommand{\bnorm}[1]{\bigg\lVert#1\bigg\rVert}
\providecommand{\snorm}[1]{\lVert#1\rVert}

\newcommand{\tnorm}[1]{{\left\vert\kern-0.25ex\left\vert\kern-0.25ex\left\vert #1 
    \right\vert\kern-0.25ex\right\vert\kern-0.25ex\right\vert}}

\providecommand{\abs}[1]{\left\lvert#1\right\rvert}
\providecommand{\sabs}[1]{\lvert#1\rvert}
\providecommand{\Babs}[1]{\bigg\lvert#1\bigg\rvert}
\providecommand{\babs}[1]{\bigg\lvert#1\bigg\rvert}

\providecommand{\prscd}{\prs{\cdot,\cdot}}
\providecommand{\prs}[1]{\left\langle #1\right\rangle}
\providecommand{\sprs}[1]{\langle #1\rangle}
\providecommand{\bprs}[1]{\bigg\langle #1\bigg\rangle}

\providecommand{\rev}{$\R$ espace vectoriel}

\providecommand{\dpar}[2]{\frac{\partial #1}{\partial #2}}

\newcommand\rst[2]{{#1}_{\restriction_{#2}}}


% Multiversioning 


\usepackage{ifthen}

\newcommand{\pl}[2]{%
	\ifthenelse{\equal{\version}{poly}}{#1}{}%
	\ifthenelse{\equal{\version}{print}}{#2}{}%
}
\newcommand{\plprof}[1]{\ifthenelse{\equal{\version}{print}}{#1}{}}
\newcommand{\sld}[1]{\ifthenelse{\equal{\version}{slide}}{#1}{}}

\newcommand{\eno}[1]{%
	\ifthenelse{\equal{\version}{eno}}{#1}{}%
}
\newcommand{\cor}[1]{%
        \ifthenelse{\equal{\version}{cor}}{
\medskip 

{\small \color{gray} #1}

\medskip 
}{}
}

\mdfdefinestyle{response}{
	leftmargin=.01\textwidth,
	rightmargin=.01\textwidth,
	linewidth=1pt
	hidealllines=false,
	leftline=true,
	rightline=true,topline=true,bottomline=true,
        skipabove=\baselineskip,%0pt,
	%innertopmargin=-5pt,
	%innerbottommargin=2pt,
	linecolor=gray,
	innerrightmargin=0pt,
	}
\providecommand{\rep}[1]{$ $ \newline \begin{mdframed}[style=response] \vspace*{#1} \end{mdframed}}

% generate breakable white space allowing students to write notes.
\newcommand*{\DivideLengths}[2]{%
  \strip@pt\dimexpr\number\numexpr\number\dimexpr#1\relax*65536/\number\dimexpr#2\relax\relax sp\relax
}

\usepackage{xparse}
\ExplSyntaxOn
\NewDocumentCommand\manyvspace { m }
 {
  \par
  \int_step_inline:nn{#1}{\phantom{.}\vspace*{1em}\goodbreak}
 }
\ExplSyntaxOff 

%\providecommand{\rep}[1]{$ $

    %\begin{mdframed}[style=response]  
	%\vspace*{\DivideLengths{#1}{3cm}cm}
	%\pagebreak[1]	
	%\vspace*{\DivideLengths{#1}{3cm}cm}
	%\pagebreak[1]		
	%\vspace*{\DivideLengths{#1}{3cm}cm}
    %\end{mdframed}%
%}
\providecommand{\repcom}[1]{\begin{mdframed}[style=response] #1 \end{mdframed}}
\providecommand{\comexo}[1]{\bigskip {\color{gray}#1}}
\providecommand{\blanc}[1]{\vspace*{#1}}
\providecommand{\myeq}[1]{\mathrel{\overset{\makebox[0pt]{\mbox{\normalfont\tiny #1}}}{=}}}
\def\redspace{\sld{\setlength{\belowdisplayskip}{0pt} \setlength{\belowdisplayshortskip}{0pt}\setlength{\abovedisplayskip}{0pt}\setlength{\abovedisplayshortskip}{0pt}}}
%------------------------------------------------------------------------------
%\DeclareUnicodeCharacter{00A0}{~}

\pdfstringdefDisableCommands{%
    %\renewcommand*{\bm}[1]{#1}%
    \renewcommand*{\R}{R}%
    % any other necessary redefinitions 
}
\makeatother




\newcommand{\va}{variable aléatoire\xspace} 
\newcommand{\vas}{variables aléatoires\xspace}
\newcommand{\evs}{événements\xspace}
\newcommand{\ev}{événement\xspace}
\newcommand{\fdr}{fonction de répartition\xspace} 
\usepackage{xspace} % gestion des espaces apres les macros

\title{TD 2 - Autour de la loi normale}

\begin{document}

\maketitle


\bigskip

\bigskip


\begin{exo}{}
	On considère une \va $X$ de densité donnée par
	\[
	f(x) = \sqrt{\frac{2}{\pi}} e^{-x^2/2} \one_{]0, \infty[}(x)\,, \quad x \in \R\,.
	\]
	On dit alors que $X$ suit une loi normale tronquée ou loi demi-normale.
	\begin{enumerate}
		\item Proposer une méthode de simulation de $X$ à l'aide de la méthode de rejet avec $Y \sim \mathcal{E}(1)$.
		
		\cor{Notons tout d'abord que $f$ est bien une densité. On note $g$ la densité de la loi exponentielle de paramètre $1$ :
		\[
		g(x) = e^{-x} \one_{[0, \infty[}(x)\,.
		\]
		On obtient alors
		\[
		\dfrac{f(x)}{g(x)}
		= \sqrt{\dfrac{2}{\pi}} e^{-\frac{x^2}{2} + x}
		= \sqrt{\dfrac{2}{\pi}} e^{-\frac{(x-1)^2 - 1}{2}}
		\leq \sqrt{\dfrac{2e}{\pi}}\,.
		\]
		On pose donc $m = \sqrt{\frac{2e}{\pi}} \approx 1.32$ et $r(x) = e^{-\frac{(x-1)^2}{2}}$. L'algorithme de rejet consiste alors à simuler $Y \sim \mathcal{E}(1)$ et $U \sim \mathcal{U}([0,1])$ et à garder $Y$ si $U \leq e^{-\frac{(Y-1)^2}{2}}$.
		}
		\item En déduire une méthode de simulation d'une loi normale.
		
		\cor{La question précédente permet de simuler une loi normale sur l'axe des réels positifs. Il reste à attribuer un signe à une telle réalisation. L'idée est de choisir aléatoirement et indépendamment le signe.
			
			On considère une \va $B$ de loi de Bernoulli de paramètre $1/2$ indépendante de $X$ simulée selon $f$ avec la question précédente. On pose alors $N = B X - (1-B) X$ qui vaut donc $X$ si $B$ vaut $1$ et $-X$ si $B$ vaut $0$. Montrons que $N$ suit une loi normale.
		
			Pour $x \in \R$, la formule des probabilités totales donne
			\begin{align*}
			\P(N \leq x)
			&= \P(N \leq x \mid B=1) \P(B=1) + \P(N \leq x \mid B=0) \P(B=0)\\
			&= \P(X \leq x \mid B=1) \dfrac{1}{2} + \P(-X \leq x \mid B=0) \dfrac{1}{2}\\\
			&= \dfrac{ \P(X \leq x) + \P(-X \leq x)}{2}\,,
			\end{align*}
			où on a utilisé l'indépendance de $X$ et $B$ dans la deuxième égalité. Par suite, si $x \leq 0$, alors la probabilité précédente devient
			\[
			\dfrac{\P(-X \leq x)}{2}
			= \dfrac{\P(X \geq -x)}{2}
			= \int_{-x}^\infty \dfrac{e^{-\frac{t^2}{2}}}{\sqrt{2 \pi}} \mathrm dt
			= \int_{-\infty}^x \dfrac{e^{-\frac{u^2}{2}}}{\sqrt{2 \pi}} \mathrm du\,,
			\]
			où la dernière égalité résulte du changement de variable $u=-t$. De même, pour $x > 0$, on obtient
			\[
			\P(N \leq x)
			= \dfrac{ \P(X \leq x) + 1}{2}
			= \int_{0}^x \dfrac{e^{-\frac{t^2}{2}}}{\sqrt{2 \pi}} \mathrm dt + \dfrac{1}{2}
			= \int_{-\infty}^x \dfrac{e^{-\frac{t^2}{2}}}{\sqrt{2 \pi}} \mathrm dt\,,
			\]
			où on a utilisé ici l'égalité $\int_{-\infty}^0 \frac{e^{-\frac{t^2}{2}}}{\sqrt{2 \pi}} = \frac{1}{2}$. Ainsi, pour tout $x \in \R$, on a
			\[
			\P(N \leq x) = \int_{-\infty}^x \dfrac{e^{-\frac{t^2}{2}}}{\sqrt{2 \pi}} \mathrm dt\,,
			\]
			ce qui prouve que $N$ suit une loi normale centrée réduite.
		}
	\end{enumerate}
\end{exo}


\begin{exo}{}
	Une \va suit une loi de Laplace de paramètre $\lambda >0$ si sa densité est donnée par
	\[
	g(x) = \dfrac{\lambda}{2} e^{-\lambda |x|}\,, \quad x \in \R\,.
	\]
	\begin{enumerate}
		\item Vérifier que cela définit bien une loi de probabilité.
		
		\cor{La fonction $g$ est mesurable, positive et vérifie
			\[
			\int_{\R} g(x) \, \mathrm dx
			= \dfrac{\lambda}{2}
			\bigg(\int_{-\infty}^0 e^{\lambda x}\, \mathrm dx
			+ \int_0^{+\infty} e^{-\lambda x}\, \mathrm dx \bigg)
			= \dfrac{1}{2}
			\bigg( \Big[e^{\lambda x}\Big]_{-\infty}^0
			- \Big[e^{\lambda x}\Big]_0^{+\infty} \bigg)
			= 1\,.
			\]
			Ainsi, $g$ est bien une densité.
		}
		\item À l'aide de la méthode de rejet, proposer une méthode pour simuler une loi normale centrée réduite à partir de $g$. Pour quelle valeur de $\lambda$ la probabilité de rejet est-elle minimale ?
		
		\cor{On procède comme dans l'exercice précédent, avec cette fois-ci
		\[
		f(x) = \dfrac{1}{\sqrt{2 \pi}} e^{-\frac{x^2}{2}}\,, \quad x \in \R\,.
		\]
		On obtient
		\[
		\dfrac{f(x)}{g(x)}
		= \dfrac{2}{\lambda \sqrt{2\pi}} e^{-\frac{x^2}{2} + \lambda |x|}
		= \dfrac{1}{\lambda} \sqrt{\dfrac{2}{\pi}} e^{-\frac{(|x|-\lambda)^2 + \lambda}{2}}
		\leq \dfrac{1}{\lambda} \sqrt{\dfrac{2}{\pi}} e^{\frac{\lambda^2}{2}}
		=: m(\lambda)\,.
		\]
		Il reste à minimiser $m(\lambda)$ ce qui revient à minimiser son logarithme. La dérivée de $\lambda \mapsto \ln (m(\lambda))$ vaut $-\frac{1}{\lambda} + \lambda$ donc s'annule en $\lambda = 1$. Le minimum vaut alors $m(1) = \sqrt{\dfrac{2 e}{\pi}}$. On retrouve la borne de l'exercice précédent.
		
		L'algorithme de rejet consiste alors à simuler une \va $U \sim \mathcal{U}([0,1])$ et $Y$ selon $g$, puis à conserver $Y$ si $U \leq e^{-\frac{(|Y| - 1)^2}{2}}$. Remarquons que $Y$ se simule sans problème avec la méthode d'inversion (sa fonction de répartition est bijective).
		
		D'après le cours, la probabilité de rejet est donnée par
		\[
		1 - \dfrac{1}{m} = 1 - \sqrt{\dfrac{\pi}{2 e}} \approx 0.24\,.
		\]
		}
	\end{enumerate}
\end{exo}


\begin{exo}{}
	On considère deux \vas $U$ et $V$ indépendantes de loi uniforme sur $[0,1]$.
	\begin{enumerate}
		\item Rappeler le principe de la méthode de Box-Müller.
		
		\cor{La méthode de Box-Müller consiste à poser
		\[
		N_1 = \sqrt{-2 \ln(U)} \cos(2\pi V) \quad \text{et} \quad  N_2 = \sqrt{-2 \ln(U)} \sin(2\pi V)\,.
		\]
		Le théorème de changement de variables en coordonnées polaires assure alors que $N_1$ et $N_2$ sont deux \vas indépendantes de loi normale centrée réduite.}
	
		\item Rappeler comment simuler un couple $(X,Y)$ de loi uniforme sur le disque unité.
		
		\cor{Il suffit de considérer le couple $(2U-1, 2V-1)$ qui suit une loi uniforme sur le pavé $[-1,1]^2$. Ensuite, on pose $(X,Y) = (2U-1, 2V-1)$ si le vecteur appartient au disque, sinon on répète l'opération. Voir l'exercice 8 du TD 1 pour davantage de détails.}
		
		\item On pose $T = X^2+Y^2$. Montrer que le couple $\big(X \sqrt{-2 \ln(T)/T}, Y \sqrt{-2\ln(T)/T}\big)$ est formée de deux \vas indépendantes de loi normales centrées réduites. On pourra étudier la loi des composantes radiale et angulaire de $(X,Y)$.
		
		\emph{Remarque : cette méthode de simulation est due à Marsaglia.}
		
		\cor{La composante radiale de $(X,Y)$ est la \va $T = X^2+Y^2$ qui vérifie
		\[
		\P(T \leq t)
		= \P( \sqrt{X^2+Y^2} \leq \sqrt t)
		= \P( (X,Y) \in \mathcal{D}_{\sqrt{t}} )\,, \quad t \geq 0\,,
		\]
		où $\mathcal{D}_{\sqrt{t}}$ désigne le disque de rayon $\sqrt t$. Comme le vecteur $(X,Y)$ suit une loi uniforme sur le disque $\mathcal{D}$ on en déduit que
		\[
		\P(T \leq t)
		= \dfrac{\text{Aire}(\mathcal{D}_{\sqrt{t}})}{\text{Aire}(\mathcal{D})}
		= \dfrac{\sqrt{t}^2 \pi}{1^2 \pi}
		= t\,.
		\]
		Ainsi, $T$ suit une loi uniforme sur $[0,1]$.
	
		D'autre part, notons $\Theta$ la \va aléatoire donnant l'angle du vecteur $(X,Y)$, c'est-à-dire
		\[
		\cos(\Theta) = \dfrac{X}{X^2+Y^2} \quad \text{et} \quad \sin(\Theta) = \dfrac{Y}{X^2+Y^2}\,.
		\]
		Cette quantité vérifie la relation
		\[
		\P( \Theta \leq \alpha )
		= \P( (X,Y) \in \mathcal{A}_\alpha)\,,
		\]
		où $\mathcal{A}_\alpha$ désigne l'ensemble des points du disque unité formant un angle inférieur à $\alpha$ avec l'axe des abscisses. Ce secteur angulaire a pour superficie $\frac{\alpha}{2}$, ce qui donne
		\[
		\P( \Theta \leq \alpha )
		= \P( (X,Y) \in \mathcal{A}_\alpha)
		= \dfrac{\frac{\alpha}{2}}{\pi}
		=  \dfrac{\alpha}{2 \pi}\,.
		\]
		On a utilisé la formule $\text{Aire}(\mathcal{A}_{\alpha, R}) = \frac{\alpha}{2} R^2$, où $\mathcal{A}_{\alpha, R}$ désigne l'ensemble des points du disque de rayon $R$ d'angle inférieur à $\alpha$.
	
		Bref, le couple proposé s'écrit
		\[
		\Big(X \sqrt{ \frac{-2\ln(T)}{T}}, Y \sqrt{\frac{-2 \ln(T)}{T}} \Big)
		= \Big(\sqrt{- 2 \ln(T)} \cos(\Theta), \sqrt{- 2 \ln(T)} \sin(\Theta) \Big)
		\]
		et a donc la même loi que celui proposé par l'algorithme de Box-Müller. Les deux marginales suivent ainsi une loi normale centrée réduite.}
		\item Comparer cette approche avec la méthode de Box-Müller.
		
		\cor{Cette approche a l'avantage de ne pas faire intervenir de fonctions trigonométriques (coûteuses en temps de calcul), mais s'appuie sur une procédure de rejet. Rappelons que la probabilité de rejet pour simuler des \vas sur le disque est de $21\%$ (voir l'exercice 8 du TD 1), ce qui est assez faible.}
	\end{enumerate}
\end{exo}



%\begin{exo}{}
%	Cet exercice sera utile pour traiter les convergences en loi de l'exercice 5.
%	\begin{enumerate}
%		\item Soit $(X_n)_{n \geq 1}$ une suite de \vas réelles qui converge en loi vers une \va $X$. Soit $f : \R \to \R$ une fonction continue. Montrer que $(f(X_n))_{n \geq 1}$ converge en loi vers $f(X)$.
%		
%		\cor{Rappelons que $X_n \to X$ en loi quand $n \to \infty$ si pour toute fonction $h : \R \to \R$ continue bornée,
%		\[
%		\E[h(X_n)] \to \E[h(X)]\,, \quad n \to \infty\,.
%		\]
%		Soit $h : \R \to \R$ continue bornée. La fonction $h \circ f$ est alors continue bornée donc l'hypothèse de convergence en loi de $X_n$ vers $X$ assure que
%		\[
%		\E[h(f(X_n))] \to \E[h(f(X))]\,, \quad n \to \infty\,,
%		\]
%		 ce qui prouve que $f(X_n)$ vers en loi vers $f(X)$.}
%		\item Soit $(X_n)_{n \geq 1}$ et $(Y_n)_{n \geq 1}$ deux suites de \vas réelles, $X$ une \va réelle et $c$ un réel, tels que
%		\[
%		X_n \xrightarrow[n \to \infty]{\mathcal{L}} X \quad \text{et} \quad
%		Y_n \xrightarrow[n \to \infty]{\mathcal{L}} c\,.
%		\]
%		On admet le lemme de Slutsky qui assure que sous ces hypothèses le couple $(X_n,Y_n)$ converge en loi vers $(X,c)$ (attention le résultat n'est plus vrai si $c$ n'est pas une constante).
%		
%		Montrer que
%		\[
%		X_n Y_n \xrightarrow[n \to \infty]{\mathcal{L}} c X\quad \text{et} \quad
%		X_n + Y_n \xrightarrow[n \to \infty]{\mathcal{L}} X + c\,.
%		\]
%		
%		\cor{On utilise le résultat de la question 1. en dimension $2$ (la généralisation ne pose pas de problème) avec $f_1(x,y) = xy$ et $f_2(x,y) = x+y$ qui sont des fonctions continues. Comme $(X_n, Y_n)$ converge en loi vers $(X,c)$, la question 1. assure que
%		\[
%		f_1(X_n, Y_n) \xrightarrow[n \to \infty]{\mathcal{L}} f_1(X,c)
%		\quad \text{et} \quad
%		f_2(X_n, Y_n) \xrightarrow[n \to \infty]{\mathcal{L}} f_2(X,c)\,,
%		\]
%		ce qui donne le résultat voulu.
%		}
%	\end{enumerate}
%\end{exo}


\begin{exo}{}
	Soit $X_1, \ldots, X_n$ des \vas iid de loi $\mathcal{N}(\mu, \sigma^2)$. Un estimateur de $\mu$ (resp. $\sigma^2$) est une quantité aléatoire basée sur l'échantillon $X_1, \ldots, X_n$ qui permet d'approcher $\mu$ (resp. $\sigma^2$). Le but de l'exercice est de proposer des estimateurs naturels de $\mu$ et $\sigma^2$ et d'étudier leurs propriétés.
	
	\begin{enumerate}
		\item \underline{Cas 1 : $\mu$ inconnue et $\sigma$ connu}
		
		On suppose que la valeur de $\mu$ est inconnue mais que celle de $\sigma$ est connue.
		\begin{enumerate}
			\item Proposer un estimateur $\bar X_n$, basé sur l'échantillon $X_1, \ldots, X_n$, permettant d'approcher $\mu$. Calculer l'espérance de cet estimateur et donner sa limite (en un sens à préciser) quand $n \to \infty$.
	
			\cor{Un estimateur "naturel" de l'espérance $\mu$ est la moyenne empirique
			\[
			\bar X_n = \dfrac{X_1 + \cdots +X_n}{n}\,.
			\]
			L'espérance de cet estimateur vaut
			\[
			\E[\bar X_n] = \dfrac{\E[X_1 + \cdots + X_n]}{n} = \dfrac{\E[X_1] + \cdots + \E[X_n]}{n} = \mu\,.
			\]
			Par ailleurs, la loi des grands nombres assure la convergence presque sûre de $\bar X_n$ vers $\mu$. Notons qu'ici on n'a pas utilisé le caractère gaussien des $X_i$.
		}
			\item Donner la loi de $\bar X_n$.
	
			\cor{Les \vas $X_1, \ldots, X_n$ sont indépendantes et de loi normale donc leur somme suit encore une loi normale. De plus, la variance de $\bar X_n$ est donnée par
			\[
			\var(\bar X_n)
			= \var \Big( \dfrac{X_1 + \cdots + X_n}{n} \Big)
			= \dfrac{1}{n} \Big( \var(X_1 )+ \cdots + \var(X_n) \Big)
			= \dfrac{\sigma^2}{n}\,.
			\]
			Ainsi,
			\[
			\bar X_n \sim \mathcal{N} \Big(\mu, \frac{\sigma^2}{n} \Big)\,.
			\]
			}
		
			\item Déterminer un intervalle $I_1(X_1, \ldots, X_n)$ qui contient $\mu$ avec probabilité $1-\alpha$, pour $\alpha > 0$.
		
			\cor{À partir de la question précédente, on en déduit que $\sqrt n (\bar X_n - \mu) / \sigma$ suit une loi normale centrée réduite. Notons $q_z$ le quantile d'ordre $z$ de la loi normale, c'est-à-dire vérifiant $\P(N \leq q_z) = z$, avec $N \sim \mathcal{N}(0,1)$. On obtient alors par symétrie de la loi normale que
			\[
			\P(-q_{1- \frac{\alpha}{2}} \leq \sqrt n (\bar X_n - \mu) / \sigma \leq q_{1- \frac{\alpha}{2}}) = 1 - \alpha\,.
			\]
			ce qui se réécrit
			\[
			\P\Big( \bar X_n - \frac{\sigma}{\sqrt n} q_{1-\frac{\alpha}{2}} \leq \mu \leq \bar X_n + \frac{\sigma}{\sqrt n} q_{1-\frac{\alpha}{2}} \Big) = 1 - \alpha\,.
			\]
			Ainsi, l'intervalle $I_1(X_1, \ldots, X_n)$, donné par
			\[
			I_1(X_1, \ldots, X_n)
			= \Big[ \bar X_n - \frac{\sigma}{\sqrt n} q_{1-\frac{\alpha}{2}}, \bar X_n + \frac{\sigma}{\sqrt n} q_{1-\frac{\alpha}{2}} \Big]\,,
			\]
			contient $\mu$ avec une probabilité de $1-\alpha$.}
		\end{enumerate}
		
		\item \underline{Cas 2 : $\sigma$ inconnu et $\mu$ connu}
		
		On suppose que la valeur de $\sigma$ est inconnue mais que celle de $\mu$ est connue. On rappelle que pour une \va de loi normale centrée réduite $X$, les moments sont donnés par
		\[
		\E[X^{2p+1}] = 0\ \quad \text{et} \quad \E[X^{2p}] = \dfrac{(2p)!}{2^p p!}\,, \quad p \in \N\,.
		\]		
		\begin{enumerate}
			\item Proposer un estimateur $\hat \sigma_n^2$, basé sur l'échantillon $X_1, \ldots, X_n$, permettant d'approcher $\sigma^2$. Calculer l'espérance et la variance de cet estimateur.
			
			\cor{Rappelons que $\sigma^2 = \var(X_1) = \E[(X_1 - \mu)^2]$. Un estimateur "naturel" de $\sigma^2$ est alors
			\[
			\hat \sigma_n^2 = \dfrac{1}{n} \sum_{k=1}^n (X_k - \mu)^2\,.
			\]
			L'espérance de $\hat \sigma_n$ vaut alors
			\[
			\E[\hat \sigma_n^2]
			= \dfrac{1}{n} \sum_{k=1}^n \E[(X_k - \mu)^2]
			= \var(X_1)
			= \sigma^2\,.
			\]
			Pour la variance on utilise la formule de Koenig-Huygens :
			\[
			\var(\hat \sigma_n^2)
			= \dfrac{1}{n} \var((X_1-\mu)^2)
			= \dfrac{1}{n} \Big( \E[(X_1-\mu)^4] - \E[(X_1-\mu)^2]^2 \Big)\,.
			\]
			à partir de la la formule donnée dans l'énoncé on obtient
			\[
			\E[(X_1-\mu)^4] 
			= \sigma^4 \E\Big[ \Big(\frac{X_1-\mu}{\sigma}\Big)^4 \Big]
			= \sigma^4 \E[N^4]
			=  \sigma^4 \dfrac{4!}{2^2\,2!}
			= 3 \sigma^4\,,
			\]
			où $N$ désigne une \va de loi normale centrée réduite. Par ailleurs, le terme $\E[(X_1-\mu)^2]^2$ correspond au carré de la variance de $X_1$, soit $\sigma^4$. On en déduit alors l'expression de la variance de $\hat \sigma_n^2$ :
			\[
			\var( \hat \sigma_n^2) = \dfrac{2 \sigma^4}{n}\,.
			\]
		}
			
			\item Quelle est la loi de $\frac{n \hat \sigma_n^2}{\sigma^2}$ ? En déduire un intervalle $I_2(X_1, \ldots, X_n)$ qui contient $\sigma$ avec probabilité $1-\alpha$, pour $\alpha > 0$.
			
			\cor{On remarque que la \va
			\[
			\dfrac{n \hat \sigma_n^2}{\sigma^2}
			= \sum_{j=1}^n \Big(\dfrac{X_j - \mu}{\sigma}\Big)^2
			\]
			est une somme de $n$ \vas indépendantes de loi normale centrée réduite. Ainsi, $\frac{n \hat \sigma_n^2}{\sigma^2}$ suit une loi du chi-deux à $n$ degrés de liberté.	
			
			Notons que la loi du chi-deux n'est pas symétrique : elle prend ses valeurs dans $[0, \infty[$ (ce qui est rassurant car $\hat \sigma_n^2$ est positif). Notons $q'_{\frac{\alpha}{2}}$ (resp. $q'_{1-\frac{\alpha}{2}}$) le quantile d'ordre $\frac{\alpha}{2}$ (resp. $1-\frac{\alpha}{2}$) de la loi du chi-deux à $n$ degrés de liberté. Alors,
			\[
			\P \Big(q'_{\frac{\alpha}{2}} \leq \dfrac{n \hat \sigma_n^2}{\sigma^2} \leq q'_{1-\frac{\alpha}{2}} \Big)
			= 1- \alpha\,.
			\]
			On prendra garde au fait que $q'_{\frac{\alpha}{2}}$ et $q'_{1-\frac{\alpha}{2}}$ ne sont pas opposés l'un de l'autre car la loi du chi-deux n'est pas symétrique. On obtient alors l'intervalle de confiance (asymétrique)
			\[
			I_2(X_1, \ldots, X_n)
			= \bigg[ \sqrt{\dfrac{n}{q'_{1-\frac{\alpha}{2}}}} \hat \sigma_n, \, \sqrt{\dfrac{n}{q'_{\frac{\alpha}{2}}}} \hat \sigma_n, \bigg]\,.
			\]
		}
		\end{enumerate}
		
		\item \underline{Cas 3 : $\mu$ et $\sigma$ inconnus}
		
		On suppose que l'espérance $\mu$ et l'écart-type $\sigma$ sont tous les deux inconnus.
		
		\begin{enumerate}
			\item Peut-on toujours utiliser $\bar X_n$ comme estimateur de $\mu$ ? Et $\hat \sigma_n^2$ comme estimateur de $\sigma^2$ ?
			
			\cor{La quantité $\bar X_n$ ne dépend que des $X_i$ donc peut toujours être utilisée comme estimateur de $\mu$. Par contre, $\sigma_n^2$ dépend de $\mu$ qui est inconnu, donc on ne peut plus utiliser cette quantité pour estimer $\sigma^2$.}
			
			\item On choisit comme estimateur de $\sigma^2$ la quantité
			\[
			S_n^2 = \dfrac{1}{n} \sum_{k=1}^n(X_k - \bar X_n)^2\,.
			\]
			On peut montrer que la \va $\frac{n S_n^2}{\sigma^2}$ suit une loi du chi-deux à $n-1$ degrés de liberté (on notera le perte d'un degré de liberté par rapport au cas précédent) et que cette \va est indépendante de $\bar X_n$ (cela semble étonnant mais ce résultat peut se montrer grâce au théorème de Cochran vu en M1).
			
			Donner la loi de
			\[
			\sqrt{n-1} \dfrac{\bar X_n -\mu}{S_n}\,.
			\]
			En déduire un intervalle $I_3(X_1, \ldots, X_n)$ qui contient $\mu$ avec probabilité $1-\alpha$, pour $\alpha > 0$.
			
			\cor{On part de l'égalité
				\[
				\sqrt{n-1} \dfrac{\bar X_n -\mu}{S_n}
				= \dfrac{\sqrt{n} \frac{\bar X_n -\mu}{\sigma}}{\sqrt{\frac{n S_n^2}{\sigma^2}} \times \frac{1}{\sqrt{n-1}}}\,.
				\]
				Cette quantité correspond au ratio d'une loi normale par une loi du chi-deux à $n-1$ degrés de liberté, les deux \vas étant indépendantes d'après l'énoncé. Le cours assure alors que $\sqrt{n-1} \frac{\bar X_n -\mu}{S_n}$ suit une loi de Student à $n-1$ degrés de liberté.
				
				En notant $\tilde q_{1-\frac{\alpha}{2}}$ le quantile d'ordre $1-\frac{\alpha}{2}$ d'une loi de Student (qui est symétrique), on en déduit l'intervalle de confiance demandé :
				\[
				I_3(X_1, \ldots, X_n)
				=  \Big[  \bar X_n - \frac{S_n}{\sqrt {n-1}} \tilde q_{1-\frac{\alpha}{2}}, \bar X_n + \frac{S_n}{\sqrt {n-1}} \tilde q_{1-\frac{\alpha}{2}} \Big]\,.
				\]
			}
		\end{enumerate}
		
		 
		
	\end{enumerate}
	
\end{exo}


\end{document}

La théorème central limite donne la convergence en loi
\[
\sqrt{n} \dfrac{\bar X_n - \mu}{\sigma} \xrightarrow[n \to \infty]{\mathcal{L}} N\,,
\]
où $N \sim \mathcal{N}(0,1)$. La loi limite de $\bar X_n$ est donc une loi normale et la vitesse de convergence est en $\sqrt n$.

	 Soit $\alpha > 0$. Déterminer un intervalle aléatoire $I_1 = I_1(X_1, \ldots, X_n)$ tel que
\[
\P(\mu \in I_1(X_1, \ldots, X_n)) \to 1- \alpha\,, \quad n \to \infty\,.
\]

\cor{Notons $q_{1-\frac{\alpha}{2}}$ le quantile d'ordre $1-\frac{\alpha}{2}$ de $N$ : $\P(N \leq q_{1-\frac{\alpha}{2}}) = 1 - \frac{\alpha}{2}$. Par symétrie de la loi normale, on obtient
	\[
	\P( -q_{1-\frac{\alpha}{2}} \leq N \leq q_{1-\frac{\alpha}{2}}) = 1 - \alpha
	\]
	(faire un dessin pour s'en convaincre). Ainsi, la convergence en loi de la question précédente se réécrit
	\[
	\P\Big( -q_{1-\frac{\alpha}{2}} \leq \sqrt{n} \dfrac{\bar X_n - \mu}{\sigma} \leq q_{1-\frac{\alpha}{2}} \Big)
	\to \P( -q_{1-\frac{\alpha}{2}} \leq N \leq q_{1-\frac{\alpha}{2}}) = 1 - \alpha\,, \quad n \to \infty\,,
	\]
	ou encore
	\[
	\P\Big( \bar X_n - \frac{\sigma}{\sqrt n} q_{1-\frac{\alpha}{2}} \leq \mu \leq \bar X_n - \frac{\sigma}{\sqrt n} q_{1-\frac{\alpha}{2}} \Big)
	\to 1 - \alpha\,, \quad n \to \infty\,,
	\]
	L'intervalle aléatoire, appelé intervalle de confiance, est donc
	\[
	I_1
	= I_1(X_1, \ldots, X_n)
	= \Big[ \bar X_n - \frac{\sigma}{\sqrt n} q_{1-\frac{\alpha}{2}}, \bar X_n + \frac{\sigma}{\sqrt n} q_{1-\frac{\alpha}{2}} \Big]\,.
	\]
}


\begin{exo}{}
	Soit $X_1, \ldots, X_n$ des \vas iid de variance $\sigma^2 < \infty$. On note $\mu$ l'espérance des $X_i$. Un estimateur de $\mu$ (resp. $\sigma^2$) est une quantité aléatoire basée sur l'échantillon $X_1, \ldots, X_n$ qui permet d'approcher $\mu$ (resp. $\sigma^2$). Le but de l'exercice est de proposer des estimateurs naturels de $\mu$ et $\sigma^2$ et d'étudier leurs propriétés.
	
	\begin{enumerate}		
		\item \underline{Cas 1 : $\mu$ inconnue et $\sigma$ connu}
		
		On suppose que la valeur de $\mu$ est inconnue mais que celle de $\sigma$ est connue.
		\begin{enumerate}
			\item Proposer un estimateur $\bar X_n$, basé sur l'échantillon $X_1, \ldots, X_n$, permettant approcher $\mu$ ? Calculer l'espérance de cet estimateur et donner sa limite (en un sens à préciser) quand $n \to \infty$.
			
			\cor{Un estimateur "naturel" de l'espérance $\mu$ est la moyenne empirique
			\[
			\bar X_n = \dfrac{X_1 + \cdots +X_n}{n}\,.
			\]
			L'espérance de cet estimateur vaut
			\[
			\E[\bar X_n] = \dfrac{\E[X_1 + \cdots + X_n]}{n} = \dfrac{\E[X_1] + \cdots + \E[X_n]}{n} = \mu\,.
			\]
			Par ailleurs, la loi des grands nombres assure la convergence presque sûre de $\bar X_n$ vers $\mu$.
			}
			\item On souhaite préciser l'approximation précédente. Donner la loi limite de $\bar X_n$.
			
			\cor{La théorème central limite donne la convergence en loi
			\[
			\sqrt{n} \dfrac{\bar X_n - \mu}{\sigma} \xrightarrow[n \to \infty]{\mathcal{L}} N\,,
			\]
			où $N \sim \mathcal{N}(0,1)$. La loi limite de $\bar X_n$ est donc une loi normale et la vitesse de convergence est en $\sqrt n$.
			}
			\item Soit $\alpha > 0$. Déterminer un intervalle aléatoire $I_1 = I_1(X_1, \ldots, X_n)$ tel que
			\[
			\P(\mu \in I_1(X_1, \ldots, X_n)) \to 1- \alpha\,, \quad n \to \infty\,.
			\]
			
			\cor{Notons $q_{1-\frac{\alpha}{2}}$ le quantile d'ordre $1-\frac{\alpha}{2}$ de $N$ : $\P(N \leq q_{1-\frac{\alpha}{2}}) = 1 - \frac{\alpha}{2}$. Par symétrie de la loi normale, on obtient
			\[
			\P( -q_{1-\frac{\alpha}{2}} \leq N \leq q_{1-\frac{\alpha}{2}}) = 1 - \alpha
			\]
			(faire un dessin pour s'en convaincre). Ainsi, la convergence en loi de la question précédente se réécrit
			\[
			\P\Big( -q_{1-\frac{\alpha}{2}} \leq \sqrt{n} \dfrac{\bar X_n - \mu}{\sigma} \leq q_{1-\frac{\alpha}{2}} \Big)
			\to \P( -q_{1-\frac{\alpha}{2}} \leq N \leq q_{1-\frac{\alpha}{2}}) = 1 - \alpha\,, \quad n \to \infty\,,
			\]
			ou encore
			\[
			\P\Big( \bar X_n - \frac{\sigma}{\sqrt n} q_{1-\frac{\alpha}{2}} \leq \mu \leq \bar X_n - \frac{\sigma}{\sqrt n} q_{1-\frac{\alpha}{2}} \Big)
			\to 1 - \alpha\,, \quad n \to \infty\,,
			\]
			L'intervalle aléatoire, appelé intervalle de confiance, est donc
			\[
			I_1
			= I_1(X_1, \ldots, X_n)
			= \Big[ \bar X_n - \frac{\sigma}{\sqrt n} q_{1-\frac{\alpha}{2}}, \bar X_n + \frac{\sigma}{\sqrt n} q_{1-\frac{\alpha}{2}} \Big]\,.
			\]
			}
		\end{enumerate}
	
		\item \underline{Cas 2 : $\sigma$ inconnu et $\mu$ connue}
		
		On suppose que la valeur de $\sigma$ est inconnue, mais que celle de $\mu$ est connue.
		\begin{enumerate}
			\item Proposer un estimateur $\hat \sigma_n^2$, basé sur l'échantillon $X_1, \ldots, X_n$, permettant approcher $\sigma^2$ ? Calculer l'espérance de cet estimateur et donner sa limite (en un sens à préciser) quand $n \to \infty$.
			
			\cor{Rappelons que $\sigma^2 = \var(X_1) = \E[(X_1 - \mu)^2]$. Un estimateur "naturel" de $\sigma^2$ est alors
			\[
			\hat \sigma_n^2 = \dfrac{1}{n} \sum_{k=1}^n (X_k - \mu)^2\,.
			\]
			L'espérance de $\hat \sigma_n$ vaut alors
			\[
			\E[\hat \sigma_n^2]
			= \dfrac{1}{n} \sum_{k=1}^n \E[(X_k - \mu)^2]
			= \var(X_1)
			= \sigma^2\,.
			\]
			La loi des grandes nombres, appliquée aux \vas iid $Y_k = (X_k - \mu)^2$ donne enfin la convergence presque sûre
			\[
			\hat \sigma_n^2 = \dfrac{1}{n} \sum_{k=1}^n (X_k - \mu)^2
			= \dfrac{1}{n} \sum_{k=1}^n Y_k
			\to \E[Y_1]
			= \E[(X_1 - \mu)^2]
			= \sigma^2\,, \quad n \to \infty\,.
			\]
		}
			\item On souhaite préciser l'approximation précédente. Donner la loi limite de $\hat \sigma_n^2$. On fera les hypothèses nécessaires sur la loi des $X_i$.
			
			\cor{Déterminons la variance de $\hat \sigma_n$. L'indépendance des $X_k$ assure que
			\[
			\var(\hat \sigma_n)
			= \dfrac{1}{n^2} \sum_{k=1}^n \var((X_k-\mu)^2)
			= \dfrac{1}{n} \var((X_1 - \mu)^2)\,.
			\]
			La formule de Koenig-Huygens donne alors
			\[
			\var((X_1 - \mu)^2)\
			= \E[(X_1 - \mu)^4] - \E[(X_1 - \mu)^2]^2\,.
			\]
			Le premier terme correspond au moment centré d'ordre $4$, notons-le $\mu_4$, tandis que le deuxième terme correspond à la variance de $X_1$ au carré, c'est-à-dire $\sigma^4$. Ainsi, la variance de $\hat \sigma_n^2$ vaut
			\[
			\var(\hat \sigma_n)
			= \dfrac{\mu_4 - \sigma^4}{n}\,.
			\]
			Notons que cette quantité existe sous réserve de l'existence du moment $\mu_4$.
			
			Le théorème central-limite assure alors que
			\[
			\dfrac{\hat \sigma_n^2 - \E[\hat \sigma_n^2]}{\sqrt{\var(\hat \sigma_n^2)}}
			= \sqrt{n} \dfrac{\hat \sigma_n^2 - \sigma^2}{\sqrt{\mu_4 - \sigma^4}}
			 \xrightarrow[n \to \infty]{\mathcal{L}} N\,,
			 \]
			 où $N \sim \mathcal{N}(0,1)$.
			}
			\item Soit $\alpha > 0$. Déterminer un intervalle aléatoire $I_2 = I_2(X_1, \ldots, X_n)$ tel que
			\[
			\P(\sigma \in I_2(X_1, \ldots, X_n)) \to 1- \alpha\,, \quad n \to \infty\,.
			\]
			
			\cor{La même méthode qu'à la question 1. (c) donne alors l'intervalle de confiance
			\[
			I_2
			= I_2(X_1, \ldots, X_n)
			= \Big[  \hat \sigma_n^2 - \frac{\sqrt{\mu_4 - \sigma^4}}{\sqrt n} q_{1-\frac{\alpha}{2}}, \hat \sigma_n^2 + \frac{\sqrt{\mu_4 - \sigma^4}}{\sqrt n} q_{1-\frac{\alpha}{2}} \Big]\,,
			\]
			où $q_{1-\frac{\alpha}{2}}$ désigne encore le quantile d'ordre $1-\frac{\alpha}{2}$ d'une loi normale centrée réduite.}
		\end{enumerate}
	
		\item \underline{Cas 3 : $\mu$ et $\sigma$ inconnus}
		
		On suppose que l'espérance $\mu$ et l'écart-type $\sigma$ sont tous les deux inconnus. On pose
		\[
		\bar X_n = \dfrac{X_1+\cdots+X_n}{n}\,,
		\quad \text{et} \quad
		S_n^2 = \dfrac{1}{n} \sum_{k=1}^n(X_k - \bar X_n)^2\,.
		\]
		\begin{enumerate}
		\item Étudier la convergence presque sûre de $\bar X_n$ et $S_n^2$.
		
		\cor{La convergence presque sûre de $\bar X_n$ vers $\mu$ est une conséquence directe de la loi des grands nombres. Concernent $S_n^2$, on décompose cette quantité via
		\[
		S_n^2
		= \dfrac{1}{n} \sum_{k=1}^n X_k^2 - 2 \bar X_n \dfrac{1}{n} \sum_{k=1}^n X_k + \bar X_n^2
		= \dfrac{1}{n} \sum_{k=1}^n X_k^2 - \bar X_n^2\,.
		\]
		On applique alors séparément la loi des grands nombres aux deux termes, ce qui donne
		\[
		S_n^2
		\to
		\E[X_1^2] - \mu^2\,, \quad n \to \infty\,.
		\]
		Il suffit alors de reamrquer que $\E[X_1^2] - \mu^2 = \var(X_1)	= \sigma^2$. Ainsi, $S_n^2$ convergence presque sûrement vers $\sigma^2$ quand $n \to \infty$.}
		\item Calculer $\E[S_n^2]$. En déduire un estimateur $s_n^2$ qui vérifie à la fois $s_n^2 \to \sigma^2$ p.s. et $\E[s_n^2] = \sigma^2$.
		
		\cor{En reprenant le calcul de la question précédente, on a l'égalité
		\[
		S_n^2
		= \dfrac{1}{n} \sum_{k=1}^n X_k^2 - \bar X_n^2\,.
		\]
		On utilise alors la formule de Koenig-Huygens sous la forme $\E[Y^2] = \var(Y^2) + \E[Y]^2$. Ainsi, l'espérance du premier terme vaut
		\[
		\E[X_1^2] = \sigma^2 + \mu^2\,,
		\]
		tandis que celle du deuxième terme vaut
		\[
		\E[\bar X_n^2]
		= \var( \bar X_n) + \E[\bar X_n]^2
		= \dfrac{\sigma}{n} + \mu^2\,,
		\]
		où on a utilisé les expressions de $ \var( \bar X_n)$ et $\E[\bar X_n]^2$ calculées dans la question 1. On en déduit que
		\[
		\E[S_n^2]
		= \sigma^2 + \mu^2 -\Big(\dfrac{\sigma}{n} - \mu^2\Big)
		= \dfrac{n-1}{n} \sigma^2\,.
		\]
		L'espérance de l'estimateur $S_n^2$ ne vaut pas $\sigma^2$. La relation précédente incite à poser
		\[
		s_n^2 = \dfrac{n}{n-1} S_n^2 =  \dfrac{1}{n-1} \sum_{k=1}^n(X_k - \bar X_n)^2\,,
		\]
		qui vérifie alors $\E[s_n^2] = \sigma^2$. Par ailleurs, la convergence presque sûre est encore vérifiée par $s_n^2$ :
		\[
		s_n^2 = \dfrac{n}{n-1} S_n^2 \to \sigma^2\,, \quad n \to \infty\,.
		\]
		}
		\item À partir de la question 1. (b) et de l'exercice 4, déterminer la loi limite de $\bar X_n$. En déduire un intervalle aléatoire $I_3 = I_3(X_1, \ldots, X_n)$ tel que
		\[
		\P(\mu \in I_3(X_1, \ldots, X_n)) \to 1- \alpha\,, \quad n \to \infty\,.
		\]
		
		\cor{La question 1. (b) donnait la convergence en loi
		\[
		\sqrt{n} \dfrac{\bar X_n - \mu}{\sigma} \xrightarrow[n \to \infty]{\mathcal{L}} N\,,
		\]
		où $N \sim \mathcal{N}(0,1)$. Cette fois-ci, $\sigma$ est inconnu donc on ne peut l'utiliser pour déterminer la loi limite de $\bar X_n$. On va le remplacer par son estimateur $S_n^2$ qui vérifie $S_n^2 \to \sigma^2$ presque sûrement (donc en probabilité) quand $n \to \infty$. Ainsi, on a les convergences
		\[
		\sqrt{n} \dfrac{\bar X_n - \mu}{\sigma} \xrightarrow[n \to \infty]{\mathcal{L}} N
		\quad \text{et} \quad
		\dfrac{\sigma}{S_n} \to 1\,, \text{ en probabilité}\,.
		\]
		L'exercice 4 assure alors que le produit vérifie
		\[
		\dfrac{\sigma}{S_n} \sqrt{n} \dfrac{\bar X_n - \mu}{\sigma}
		= \sqrt{n} \dfrac{\bar X_n - \mu}{S_n}
		\xrightarrow[n \to \infty]{\mathcal{L}} 1 \times N = N\,.
		\]
		Bref, on peut remplacer la quantité inconnue $\sigma$ par l'estimateur $S_n$ et la convergence reste vérifiée. L'intervalle $I_3$ s'obtient alors comme dans la question 1 :
		\[
		I_3
		= I_3(X_1, \ldots, X_n)
		= \Big[ \bar X_n - \frac{S_n}{\sqrt n} q_{1-\frac{\alpha}{2}}, \bar X_n + \frac{S_n}{\sqrt n} q_{1-\frac{\alpha}{2}} \Big]\,.
		\]
		Notons qu'on peut tout aussi bien utiliser $s_n$ plutôt que $S_n$.
		}
		\end{enumerate}
	\end{enumerate}
\end{exo}
