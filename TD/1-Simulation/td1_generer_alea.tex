\documentclass[11pt]{td_um}

%------------------------------
\def\version{cor}
%\def\version{eno}
%------------------------------

\makeatletter
%--------------------------------------------------------------------------------

\usepackage[T1]{fontenc}
\usepackage[utf8]{inputenc}
\usepackage[french]{babel}

\usepackage{lmodern}
\usepackage{dsfont}

%\usepackage{qrcode}

\usepackage{eurosym}
\usepackage{diagbox}
\usepackage{enumitem}
%\def\labelitemi{\small \textbullet}
\setlist[itemize,1]{label={\color{gray}\small \textbullet}}
\usepackage{multicol}

\usepackage{graphicx}
\usepackage{float}

\usepackage{array}
\newcolumntype{L}[1]{>{\raggedright\let\newline\\\arraybackslash\hspace{0pt}}m{#1}}
\newcolumntype{C}[1]{>{\centering\let\newline\\\arraybackslash\hspace{0pt}}m{#1}}
\newcolumntype{R}[1]{>{\raggedleft\let\newline\\\arraybackslash\hspace{0pt}}m{#1}}
%----------
%  Version
%-----------

\usepackage{fancyhdr} 
\usepackage{stmaryrd}

%--------
%  Tkz  
%--------
\usepackage{blkarray}
\usepackage[babel=true,kerning=true]{microtype}
\usepackage[caption=false]{subfig}
\usepackage{xcolor,colortbl}
\definecolor{dgreen}{RGB}{0,100,0}
\definecolor{linkcol}{RGB}{0,118,155}
\definecolor{astral}{RGB}{46,116,181}

\usepackage{diagbox,calc,soul,graphicx}

\usepackage{tikz}
\usetikzlibrary{3d,calc,fadings,decorations.pathreplacing,matrix,arrows,decorations.text}
\usetikzlibrary{patterns}
\usetikzlibrary{positioning}
\usetikzlibrary{babel}
\usetikzlibrary{shapes}
\usetikzlibrary{shadings}
\usetikzlibrary{cd}
\usepackage{tikz-3dplot}
\usepackage{pgfplots}
\usepgfplotslibrary{fillbetween}
\pgfplotsset{compat=newest}
\usepgfplotslibrary{external} 
\tikzexternalize[prefix=output_latex/]
\usepgfplotslibrary{fillbetween}
%\graphicspath{{./output-latex/}}


%\usepackage{etex}
%\reserveinserts{28}
%\usepackage{pstricks}
%\usepackage{pst-solides3d}

\usepackage{gnuplot-lua-tikz}

\newcommand\chideux[1]{#1<=0||(#1!=int(#1))?1/0:x<=0?0.0:exp((0.5*#1-1.0)*log(x)-0.5*x-lgamma(0.5*#1)-#1*0.5*log(2))}
\newcommand\gauss[2]{1/(#2*sqrt(2*pi))*exp(-((x-#1)^2)/(2*#2^2))} 
\newcommand\student[1]{gamma(.5*(#1+1))/(sqrt(#1*pi)*gamma(.5*#1))*((1+x^2/#1)^(-.5*(#1+1)))}

% display dices
\usepackage{xparse}\usetikzlibrary{shapes}
\NewDocumentCommand{\drawdie}{O{}m}{%
\begin{tikzpicture}[x=1em,y=1em,radius=0.1,#1,baseline=0.575ex]
		\draw[rounded corners=0.5] (0,0) rectangle (1,1);
	  \ifodd#2
      \fill[] (0.5,0.5) circle;
        \fi
  \ifnum#2>1
      \fill[] (0.2,0.2) circle;
          \fill[] (0.8,0.8) circle;
     \ifnum#2>3
          \fill[] (0.2,0.8) circle;
       \fill[] (0.8,0.2) circle;
           \ifnum#2>5
         \fill[] (0.8,0.5) circle;
       \fill[] (0.2,0.5) circle;
            \ifnum#2>7
           \fill[] (0.5,0.8) circle;
          \fill[] (0.5,0.2) circle;
        \fi
    \fi
      \fi
      \fi
\end{tikzpicture}%
} 
\NewDocumentCommand{\tdrawdie}{O{}m}{%
    \begin{tikzpicture}[x=.6em,y=.6em,radius=0.1,#1,baseline=0.575ex]
		\draw[rounded corners=0.5] (0,0) rectangle (1,1);
	  \ifodd#2
      \fill[] (0.5,0.5) circle;
        \fi
  \ifnum#2>1
      \fill[] (0.2,0.2) circle;
          \fill[] (0.8,0.8) circle;
     \ifnum#2>3
          \fill[] (0.2,0.8) circle;
       \fill[] (0.8,0.2) circle;
           \ifnum#2>5
         \fill[] (0.8,0.5) circle;
       \fill[] (0.2,0.5) circle;
            \ifnum#2>7
           \fill[] (0.5,0.8) circle;
          \fill[] (0.5,0.2) circle;
        \fi
    \fi
      \fi
      \fi
\end{tikzpicture}%
}

\usetikzlibrary{lindenmayersystems}
\pgfdeclarelindenmayersystem{cantor set}{
  \rule{F -> FfF}
  \rule{f -> fff}
}

%------------------
% Math environment
%------------------

\usepackage{latexsym}
\usepackage{amsmath}
\usepackage{amsbsy}
\usepackage{amsfonts}
\usepackage{amssymb}
\usepackage{nicefrac}
\usepackage{amscd}
\usepackage{amsthm}
\usepackage{mathtools}

\newtheoremstyle{definitionSs}{\topsep}{\topsep}%
     {}%         Body font
     {}%         Indent amount (empty = no indent, \parindent = para indent)
     {\sffamily\bfseries}% Thm head font
     {.}%        Punctuation after thm head
     { }%     Space after thm head (\newline = linebreak)
     {\thmname{#1}\thmnumber{~#2}\thmnote{~#3}}%         Thm head spec

\newtheoremstyle{plainSs}{\topsep}{\topsep}%
     {\itshape}%         Body font
     {}%         Indent amount (empty = no indent, \parindent = para indent)
     {\sffamily\bfseries}% Thm head font
     {.}%        Punctuation after thm head
     { }%     Space after thm head (\newline = linebreak)
     {\thmname{#1}\thmnumber{~#2}\thmnote{~#3}}%         Thm head spec

\theoremstyle{definitionSs}
\newtheorem{remark}{Remarque}[section]
\newtheorem*{remark*}{Remarque}
\newtheorem{example}{Exemple}[section]


%\usepackage[framemethod=tikz]{mdframed}
\usepackage[]{mdframed}

\newmdtheoremenv[
hidealllines=true,
leftline=true,
skipabove=0pt,
innertopmargin=-5pt,
innerbottommargin=2pt,
linewidth=4pt,
linecolor=gray!90,
innerrightmargin=0pt,
]{definition}{Définition}[section]

\newmdtheoremenv[
hidealllines=true,
leftline=true,
skipabove=0pt,
innertopmargin=-5pt,
innerbottommargin=2pt,
linewidth=4pt,
linecolor=gray!40,
innerrightmargin=0pt,
]{lemme}{Lemme}[section]

\newmdtheoremenv[
hidealllines=true,
leftline=true,
skipabove=0pt,
innertopmargin=-5pt,
innerbottommargin=2pt,
linewidth=4pt,
linecolor=gray!40,
innerrightmargin=0pt,
]{proposition}{Proposition}[section]

\newmdtheoremenv[
hidealllines=true,
leftline=true,
skipabove=0pt,
innertopmargin=-5pt,
innerbottommargin=2pt,
linewidth=4pt,
linecolor=gray!40,
innerrightmargin=0pt,
]{corollaire}{Corollaire}[section]

\newmdtheoremenv[
hidealllines=true,
leftline=true,
skipabove=0pt,
innertopmargin=-5pt,
innerbottommargin=2pt,
linewidth=4pt,
linecolor=gray!90,
innerrightmargin=0pt,
]{propdef}{Définition - Proposition}[section]

\newmdtheoremenv[
hidealllines=true,
leftline=true,
skipabove=0pt,
innertopmargin=-5pt,
innerbottommargin=2pt,
linewidth=4pt,
linecolor=gray!100,
innerrightmargin=0pt,
]{theorem}{Théorème}[section]

%---------------
% Mise en page
%--------------

\setlength{\parindent}{0pt}

\renewcommand*{\descriptionlabel}[1]{\hspace\labelsep{\sffamily #1}}
\providecommand{\defemph}[1]{{\sffamily\bfseries\color{astral}#1}}
\renewcommand{\emph}[1]{{\sffamily #1}}

%\usepackage{titlesec, blindtext, color}
%\newcommand{\hsp}{\hspace{20pt}}
%\titleformat{\chapter}[hang]{\sffamily\Huge\bfseries}{\thechapter\hsp\textcolor{gray75}{|}\hsp}{0pt}{\Huge\bfseries}
\usepackage{sectsty}
\allsectionsfont{\color{astral}\normalfont\sffamily\bfseries}

\usepackage[subfigure]{tocloft}
\renewcommand{\cfttoctitlefont}{\sffamily\bfseries\Huge\color{astral}}
\renewcommand{\cftchapfont}{\sffamily\bfseries\color{astral}}
\renewcommand{\cftsecfont}{\sffamily\bfseries}
\renewcommand{\cftsubsecfont}{\sffamily}
\usepackage{hyperref}
\hypersetup{linktocpage, colorlinks=true, linkcolor=gray, urlcolor=linkcol, citecolor=gray}


%----------------
% Some commands
%----------------

\makeatletter
\newcommand\RedeclareMathOperator{%
  \@ifstar{\def\rmo@s{m}\rmo@redeclare}{\def\rmo@s{o}\rmo@redeclare}%
}
% this is taken from \renew@command
\newcommand\rmo@redeclare[2]{%
  \begingroup \escapechar\m@ne\xdef\@gtempa{{\string#1}}\endgroup
  \expandafter\@ifundefined\@gtempa
     {\@latex@error{\noexpand#1undefined}\@ehc}%
     \relax
  \expandafter\rmo@declmathop\rmo@s{#1}{#2}}
% This is just \@declmathop without \@ifdefinable
\newcommand\rmo@declmathop[3]{%
  \DeclareRobustCommand{#2}{\qopname\newmcodes@#1{#3}}%
}
\@onlypreamble\RedeclareMathOperator
\makeatother

\newcommand{\skipline}{\vspace{\baselineskip}}
\newcommand{\noi}{\noindent}


%----- Raccourci proba ------------------
\newcommand{\Om}{\Omega}
\newcommand{\Tribu}{\cF}
\newcommand{\Tribubis}{\cG}
\newcommand{\Partie}{\cE}
\newcommand{\Proba}{\bbP}
\newcommand{\Espe}{\bbE}
\newcommand{\OF}{(\Om,\Tribu)}
\newcommand{\RBR}{\big(\bbR,\cB(\bbR)\big)}
\newcommand{\OFP}{(\Om,\Tribu,\Proba)}
\newcommand{\POm}{\cP(\Om)}
\newcommand{\Leb}{\lambda}
%----- Lettres -------------------------
\newcommand{\bbC}{\mathbb{C}}
\newcommand{\bbE}{\mathbb{E}}
\newcommand{\bbF}{\mathbb{F}}
\newcommand{\bbG}{\mathbb{G}}
\newcommand{\bbK}{\mathbb{K}}
\newcommand{\bbL}{\mathbb{L}}
\newcommand{\bbN}{\mathbb{N}}
\newcommand{\bbP}{\mathbb{P}}
\newcommand{\bbQ}{\mathbb{Q}}
\newcommand{\bbR}{\mathbb{R}}
\newcommand{\bbS}{\mathbb{S}}
\newcommand{\bbT}{\mathbb{T}}
\newcommand{\bbU}{\mathbb{U}}
\newcommand{\bbZ}{\mathbb{Z}}

\newcommand{\bfC}{\mathbf{C}}
\newcommand{\bfE}{\mathbf{E}}
\newcommand{\bfL}{\mathbf{L}}
\newcommand{\bfN}{\mathbf{N}}
\newcommand{\bfP}{\mathbf{P}}
\newcommand{\bfQ}{\mathbf{Q}}
\newcommand{\bfR}{\mathbf{R}}
\newcommand{\bfZ}{\mathbf{Z}}

\newcommand{\calA}{\mathcal{A}}
\newcommand{\calB}{\mathcal{B}}
\newcommand{\calC}{\mathcal{C}}
\newcommand{\calE}{\mathcal{E}}
\newcommand{\calF}{\mathcal{F}}
\newcommand{\calG}{\mathcal{G}}
\newcommand{\calJ}{\mathcal{J}}
\newcommand{\calL}{\mathcal{L}}
\newcommand{\calM}{\mathcal{M}}
\newcommand{\calN}{\mathcal{N}}
\newcommand{\calO}{\mathcal{O}}
\newcommand{\calP}{\mathcal{P}}
\newcommand{\calS}{\mathcal{S}}
\newcommand{\calT}{\mathcal{T}}
\newcommand{\calV}{\mathcal{V}}
\newcommand{\calX}{\mathcal{X}}

\newcommand{\fA}{\mathfrak{A}}
\newcommand{\fB}{\mathfrak{B}}
\newcommand{\fF}{\mathfrak{F}}

% la droite réelle achevée
\newcommand{\barR}{\overline{\mathbb{R}}}

%
%------------------------------- Majuscules calligraphiques
\renewcommand{\geq}{\geqslant}
\renewcommand{\ge}{\geqslant}
\renewcommand{\leq}{\leqslant}
\renewcommand{\le}{\leqslant}
%
\usepackage{mathrsfs}
\newcommand{\scrL}{\mathscr{L}}
\newcommand{\scrF}{\mathscr{F}}
\newcommand{\scrM}{\mathscr{M}}
\newcommand{\scrD}{\mathscr{D}}
\newcommand{\scrE}{\mathscr{E}}
\newcommand{\scrA}{\mathscr{A}}
\newcommand{\scrB}{\mathscr{B}}
\newcommand{\scrC}{\mathscr{C}}
\newcommand{\scrI}{\mathscr{I}}
\newcommand{\scrJ}{\mathscr{J}}
\newcommand{\scrK}{\mathscr{K}}
\newcommand{\scrS}{\mathscr{S}}
\newcommand{\scrO}{\mathscr{O}}
\newcommand{\scrP}{\mathscr{P}}
\newcommand{\scrR}{\mathscr{R}}
\newcommand{\scrT}{\mathscr{T}}
\newcommand{\scrZ}{\mathscr{Z}}
%
\newcommand{\vois}{\mathscr{V}}
\newcommand{\ouv}{\mathscr{O}}
\newcommand{\ferm}{\mathscr{F}}

%
%------------------------------- Intervalles
%
\newcommand{\oo}[1]{\mathopen{]}#1\mathclose{[}}
\newcommand{\fo}[1]{\mathopen{[}#1\mathclose{[}}
\newcommand{\of}[1]{\mathopen{]}#1\mathclose{]}}
\newcommand{\ff}[1]{\mathopen{[}#1\mathclose{]}}
\newcommand{\lbb}{[\![}
\newcommand{\rbb}{]\!]}

%
\newcommand{\ex}{e}
\newcommand{\ind}{\mathrm{Ind}}
\newcommand{\spt}{\mathrm{Spt}}
\newcommand{\pd}{\frac{\pi}{2}}
\newcommand{\E}{\mathrm{E}}
\newcommand{\diff}{\mathrm{Diff}}
\newcommand{\diam}{\mathrm{diam}}
\newcommand{\comp}[1]{#1^\mathrm{c}}
\newcommand{\vvv}{|\!|\!|}
\newcommand{\dd}[2]{\frac{\partial #1}{\partial #2}}
%
%----- chapeaux, tildes -------------------------
\newcommand{\wb}[1]{\overline{#1}}
\newcommand{\wh}[1]{\widehat{#1}}
\newcommand{\wt}[1]{\widetilde{#1}}

%----- ccouleurs -------------------------
\newcommand{\rose}[1]{\textcolor{magenta}{#1}}

%----- Maths -----------------------
\newcommand {\expl}[2]{\mathrm{e}^{-\Lambda(#1,#2)}}
\newcommand {\expla}[2]{\mathrm{e}^{-\alpha #2-\Lambda(#1,#2)}}
\newcommand {\Li}[1]{[ #1 ]}
%\newcommand{\1}{\mathbbm{1}}
\newcommand {\tstar}[1]{t^{*}(#1)}
\DeclareMathOperator{\aire}{Aire}
\providecommand{\gf}{g\circ f}
\providecommand{\R}{\ensuremath \mathbb{R}}
\providecommand{\C}{\ensuremath \mathbb{C}}
\providecommand{\reg}[1]{\mathcal{C}^{#1}}
\providecommand{\N}{\mathbb{N}}
\providecommand{\M}{\mathcal{M}}
\providecommand{\Q}{\mathbb{Q}}
\renewcommand{\L}{\mathcal{L}}
\providecommand{\D}{\mathcal{D}}
\providecommand{\Cc}{\mathcal{C}}
\providecommand{\F}{\mathcal{F}}
\providecommand{\Ee}{\mathcal{E}}
\providecommand{\G}{\mathcal{G}}
\providecommand{\Z}{\mathbb{Z}}
\providecommand{\x}{\ensuremath\boldsymbol{x}}
\providecommand{\y}{\ensuremath\boldsymbol{y}}
\providecommand{\1}{\mathds{1}}
\providecommand{\p}{\partial}
\providecommand{\Pp}{\mathcal{P}}
\providecommand{\P}{\mathbb{P}}
\providecommand{\E}{\mathbb{E}}
\providecommand{\U}{\mathcal{U}}
\providecommand{\V}{\mathcal{V}}
\providecommand{\ie}{\textit{i.e. }}
\renewcommand{\P}{\mathbb{P}}
\renewcommand{\S}{\mathcal{S}}
\providecommand{\E}{\mathbb{E}}
\providecommand{\one}{\mathds{1}}
\DeclareMathOperator{\card}{Card}
\DeclareMathOperator{\vol}{Vol}
\DeclareMathOperator{\var}{Var}
\DeclareMathOperator{\vect}{\mathsf{Vect}}
\DeclareMathOperator{\med}{median}
\DeclareMathOperator{\hess}{Hess}
\DeclareMathOperator{\jac}{Jac}
\DeclareMathOperator{\cov}{cov}
\DeclareMathOperator{\diag}{\mathsf{Diag}}
\DeclareMathOperator{\im}{\mathsf{Im}}
\DeclareRobustCommand{\re}{\mathsf{Re}}
\RedeclareMathOperator{\ker}{\mathsf{Ker}}
\RedeclareMathOperator{\det}{\mathsf{det}}
\DeclareMathOperator{\rank}{\mathsf{rank}}
\DeclareMathOperator{\tr}{\mathsf{tr}}
\DeclareMathOperator{\id}{\mathsf{Id}}
\DeclareMathOperator{\can}{\mathsf{can}}
\DeclareMathOperator{\com}{\mathsf{com}}
\DeclareMathOperator*{\argmax}{Argmax}
\providecommand{\B}{\mathsf{B}}

\providecommand{\ncd}{\norm{\cdot}}
\providecommand{\norm}[1]{\left\lVert#1\right\rVert}
\providecommand{\bnorm}[1]{\bigg\lVert#1\bigg\rVert}
\providecommand{\snorm}[1]{\lVert#1\rVert}

\newcommand{\tnorm}[1]{{\left\vert\kern-0.25ex\left\vert\kern-0.25ex\left\vert #1 
    \right\vert\kern-0.25ex\right\vert\kern-0.25ex\right\vert}}

\providecommand{\abs}[1]{\left\lvert#1\right\rvert}
\providecommand{\sabs}[1]{\lvert#1\rvert}
\providecommand{\Babs}[1]{\bigg\lvert#1\bigg\rvert}
\providecommand{\babs}[1]{\bigg\lvert#1\bigg\rvert}

\providecommand{\prscd}{\prs{\cdot,\cdot}}
\providecommand{\prs}[1]{\left\langle #1\right\rangle}
\providecommand{\sprs}[1]{\langle #1\rangle}
\providecommand{\bprs}[1]{\bigg\langle #1\bigg\rangle}

\providecommand{\rev}{$\R$ espace vectoriel}

\providecommand{\dpar}[2]{\frac{\partial #1}{\partial #2}}

\newcommand\rst[2]{{#1}_{\restriction_{#2}}}


% Multiversioning 


\usepackage{ifthen}

\newcommand{\pl}[2]{%
	\ifthenelse{\equal{\version}{poly}}{#1}{}%
	\ifthenelse{\equal{\version}{print}}{#2}{}%
}
\newcommand{\plprof}[1]{\ifthenelse{\equal{\version}{print}}{#1}{}}
\newcommand{\sld}[1]{\ifthenelse{\equal{\version}{slide}}{#1}{}}

\newcommand{\eno}[1]{%
	\ifthenelse{\equal{\version}{eno}}{#1}{}%
}
\newcommand{\cor}[1]{%
        \ifthenelse{\equal{\version}{cor}}{
\medskip 

{\small \color{gray} #1}

\medskip 
}{}
}

\mdfdefinestyle{response}{
	leftmargin=.01\textwidth,
	rightmargin=.01\textwidth,
	linewidth=1pt
	hidealllines=false,
	leftline=true,
	rightline=true,topline=true,bottomline=true,
        skipabove=\baselineskip,%0pt,
	%innertopmargin=-5pt,
	%innerbottommargin=2pt,
	linecolor=gray,
	innerrightmargin=0pt,
	}
\providecommand{\rep}[1]{$ $ \newline \begin{mdframed}[style=response] \vspace*{#1} \end{mdframed}}

% generate breakable white space allowing students to write notes.
\newcommand*{\DivideLengths}[2]{%
  \strip@pt\dimexpr\number\numexpr\number\dimexpr#1\relax*65536/\number\dimexpr#2\relax\relax sp\relax
}

\usepackage{xparse}
\ExplSyntaxOn
\NewDocumentCommand\manyvspace { m }
 {
  \par
  \int_step_inline:nn{#1}{\phantom{.}\vspace*{1em}\goodbreak}
 }
\ExplSyntaxOff 

%\providecommand{\rep}[1]{$ $

    %\begin{mdframed}[style=response]  
	%\vspace*{\DivideLengths{#1}{3cm}cm}
	%\pagebreak[1]	
	%\vspace*{\DivideLengths{#1}{3cm}cm}
	%\pagebreak[1]		
	%\vspace*{\DivideLengths{#1}{3cm}cm}
    %\end{mdframed}%
%}
\providecommand{\repcom}[1]{\begin{mdframed}[style=response] #1 \end{mdframed}}
\providecommand{\comexo}[1]{\bigskip {\color{gray}#1}}
\providecommand{\blanc}[1]{\vspace*{#1}}
\providecommand{\myeq}[1]{\mathrel{\overset{\makebox[0pt]{\mbox{\normalfont\tiny #1}}}{=}}}
\def\redspace{\sld{\setlength{\belowdisplayskip}{0pt} \setlength{\belowdisplayshortskip}{0pt}\setlength{\abovedisplayskip}{0pt}\setlength{\abovedisplayshortskip}{0pt}}}
%------------------------------------------------------------------------------
%\DeclareUnicodeCharacter{00A0}{~}

\pdfstringdefDisableCommands{%
    %\renewcommand*{\bm}[1]{#1}%
    \renewcommand*{\R}{R}%
    % any other necessary redefinitions 
}
\makeatother




\newcommand{\va}{variable aléatoire\xspace}
\newcommand{\vas}{variables aléatoires\xspace}
\newcommand{\evs}{événements\xspace}
\newcommand{\ev}{événement\xspace}
\newcommand{\fdr}{fonction de répartition\xspace}
\usepackage{xspace} % gestion des espaces apres les macros
\usepackage[ruled]{algorithm2e} % to write algorithms

\title{TD 1 - Simulation de variables aléatoires}

\begin{document}

\maketitle


\bigskip

\bigskip



\begin{exo}{}
	On dispose d'un dé à 6 faces. Proposer une méthode pour simuler le résultat d'un Pile ou Face. On détaillera la \va utilisée.

	\cor{Il suffit d'associer Pile aux résultats pairs et Face aux résultats impairs. Précisons ceci.

	Soit $\Omega = \{1, 2, 3, 4, 5, 6\}$ l'ensemble des résultats du dé que l'on munit de la tribu pleine $\mathcal{F} = \mathcal{P}(\Omega)$ et de la probabilité uniforme $\P$. On dispose alors d'un espace probabilisé $(\Omega, \mathcal{F}, \P)$. On considère également l'ensemble $\Omega' = \{\text{Pile}, \text{Face}\}$ que l'on munit de la tribu pleine $\mathcal{F}' = \mathcal{P}(\Omega')$. On définit alors la \va $X: \Omega \to \Omega'$ par
	\[
	X(k) =
	\bigg\{ \begin{array}{ll}
		\text{Pile} & \mbox{si $k$ est pair,} \\
		\text{Face} & \mbox{si $k$ est impair.}
	\end{array}
	\]
	La loi de $X$ est alors la mesure image de $\P$ par la \va $X$. Cette mesure est définie par
	\[
	\P(X = \text{Pile}) = \P(\{2,4,6\}) = \dfrac{1}{2}
	\quad \text{et} \quad
	\P(X = \text{Face}) = \P(\{1,3,5\}) = \dfrac{1}{2}\,.
	\]
	La mesure image correspond encore à la probabilité uniforme, cette fois sur $\Omega'$. On obtient bien la résultat d'un Pile ou Face.
	}
\end{exo}


\begin{exo}{}
	On dit qu'une \va $X$ suit une loi de Cauchy si $X$ admet une densité $f$ par rapport à la mesure de Lebesgue sur $\R$ définie par
	\[
	f(x) = \dfrac{1}{\pi(1+x^2)}\,, \quad x \in \R\,.
	\]
	\begin{enumerate}
		\item Montrer que cela définit bien une loi de probabilité.

		\cor{La fonction $f$ est mesurable, positive et vérifie
		\[
		\int_\R f(x)\, \mathrm d x
		= \int_{-\infty}^{+\infty} \dfrac{1}{\pi(1+x^2)} \, \mathrm dx
		= \dfrac{1}{\pi} \big[\arctan(x)\big]_{-\infty}^{+\infty}
		= \dfrac{1}{\pi} \Big[\dfrac{\pi}{2} - \Big(-\dfrac{\pi}{2}\Big)\Big]
		= 1\,.
		\]
		C'est donc bien une densité.}
		\item En utilisant la méthode d'inversion, donner un moyen de simuler cette loi.

		\cor{Un calcul similaire donne la \fdr $F_X$ de $X$:
		\[
		F_X(x)
		= \int_{-\infty}^x \dfrac{1}{\pi(1+t^2)} \, \mathrm dt
		= \dfrac{\arctan(x)}{\pi} + \dfrac{1}{2}\,, \quad x \in \R\,.
		\]
		La fonction $F_X$ est donc bijective d'inverse
		\[
		F_X^{-1}(u) = \tan\Big(\pi\Big(u - \frac{1}{2}\Big)\Big)\,.
		\]
		Ainsi, si $U$ suit une loi uniforme sur $[0,1]$, alors $\tan(\pi(U- 1/2))$ suit une loi de Cauchy. Cela donne une interprétation géométrique de la loi de Cauchy: c'est la tangente d'une angle tiré uniformément entre $-\pi/2$ et $\pi/2$.
		}
	\end{enumerate}
\end{exo}


\begin{exo}{}
	On considère une \va $X$ suivant une loi géométrique de paramètre $p \in ]0,1[$.
	\begin{enumerate}
		\item Rappeler la loi de $X$ et son interprétation en terme de temps d'attente.

		\cor{La loi de $X$ correspond à la mesure de probabilité
		\[
		\P_X = \sum_{n=1}^\infty p(1-p)^{n-1} \delta_n\,.
		\]
		Autrement dit, $\P(X=n) = p(1-p)^{n-1}$ pour tout $n \in \N^*$. Cette loi modélise le temps d'attente avant le premier succès lorsque l'on répète de manière indépendante une expérience de Bernoulli ayant une probabilité de succès $p$.}

    \item Comment obtenir une loi géométrique à partir d'une suite $(B_n)_{n \geq 1}$ de \vas \textit{i.i.d.} de loi de Bernoulli de paramètre $p$ ?

		\cor{En suivant l'interprétation en terme de temps d'attente, on définit
		\[
		T = \inf \{n \in \N^*: B_n =1\}\,.
		\]
		Montrons que $T$ suit une loi géométrique de paramètre $p$. Pour $n \in \N^*$, l'\ev $\{T=n\}$ est réalisé si et seulement si $B_1 = \cdots = B_{n-1} = 0$ et $B_n =1$. On en déduit alors par indépendance des $B_k$ que
		\begin{align*}
		\P(T=n)
		&= \P(B_1 = \cdots = B_{n-1} = 0, B_n =1)\\
		&= \P(B_1 = 0) \cdots \P(B_{n-1} = 0) \P(B_n =1)\\
		&= (1-p)^{n-1} p\,,
		\end{align*}
		ce qui prouve que $T$ suit une loi géométrique de paramètre $p$.}
		\item En déduire une méthode pour simuler une loi géométrique à partir d'une suite $(U_n)_{n \geq 1}$ de \vas uniformes.

		\cor{On rappelle que si $U$ suit une loi uniforme sur $[0,1]$, alors la \va $B = \one_{\{U \leq p\}}$ suit une loi de Bernoulli de paramètre $p$. Par suite, la \va
		\[
		T
		= \inf \{n \in \N^*: \one_{\{U_n \leq p\}} =1\}
		= \inf \{n \in \N^*: U_n \leq p\}
		\]
		suit une loi géométrique de paramètre $p$.}
		\item Soit $E$ une \va de loi exponentielle de paramètre $\lambda > 0$. On considère la partie entière supérieure $Y = \lceil E \rceil$. Déterminer la loi de $Y$. En déduire un autre moyen de simuler une loi géométrique de paramètre $p$ à partir d'une \va uniforme $U$.

		\cor{La \va $E$ prend ses valeurs dans $]0,\infty[$ donc $Y$ prend ses valeurs dans $\N^*$. De plus,
		\[
		\P(Y = n)
		= \P(n-1 < E \leq n)
		= \int_{n-1}^n \lambda e^{-\lambda t} \, \mathrm dt
		= e^{-\lambda(n-1)} - e^{-\lambda n}
		= (e^{-\lambda})^{n-1} (1- e^{-\lambda})\,.
		\]
		Ainsi, $Y$ suit une loi géométrique de paramètre $p = 1-e^{-\lambda}$.

		On simule alors une loi exponentielle de paramètre $\lambda = -\ln(1-p)$ avec la méthode d'inversion: $E = - \ln(U)/\lambda$. Par suite, la \va $\lceil E \rceil$ suit une loi géométrique de paramètre $p$.
	}
	\end{enumerate}
\end{exo}


\begin{exo}{}
	La loi Beta de paramètres $\alpha > 1$, $\beta > 1$, notée $\text{Beta}(\alpha, \beta)$, est donnée par la densité
	\[
	f_{\alpha, \beta}(x)
	= \dfrac{1}{B(\alpha,\beta)} x^{\alpha-1} (1-x)^{\beta-1}\,,
	\quad x \in [0,1]\,,
	\]
	où $B$ désigne la fonction beta définie par
	\[
	B (x,y) = \int_0^1 t^{x-1} (1-t)^{y-1} dt\,.
	\]
	À l'aide de la méthode de rejet, construisez une algorithme permettant de simuler $n$ \vas de loi $\text{Beta}(\alpha,\beta)$.

	\cor{On remarque que la densité vérifie $f_{\alpha, \beta}(x) \leq m \one_{[0,1]}(x)$, où $m = \frac{\Gamma(\alpha+\beta)}{\Gamma(\alpha) \Gamma(\beta)}$. On peut donc appliquer la méthode de rejet avec $m$ et la densité uniforme. L'algorithme est alors le suivant

		\begin{algorithm}[H]
			\SetAlgoLined
			\textbf{Entrée:} une taille d'échantillon $n$

			$k \leftarrow 1$

			\textbf{Tant que} $k \leq n$:
			\begin{enumerate}
				\item Générer $U,\, Y \sim \mathcal{U}([0,1])$ indépendantes
				\item
				\begin{itemize}[label=$\bullet$]
					\item Si $U \leq Y^{\alpha-1}(1-Y)^{\beta-1}$, alors $X_k \leftarrow Y$ et $k \leftarrow k+1$,
					\item Sinon rejeter et revenir en 1.
				\end{itemize}
			\end{enumerate}
			\textbf{Sortie:} $X_1, \ldots, X_n$
			\caption{Simulation de $n$ \vas de loi Beta.}
		\end{algorithm}
%--------------------------------------------
                \begin{center}
                    \begin{tikzpicture}[scale=1]
                        \begin{axis}[,xlabel=$Y$,
                                ylabel={$U$},
                                width=6cm,
                                height=6cm,
                                ymin=0,
                                xmin=0,
                                ymax=1,
                                xmax=1,
                                grid=major,
                                legend style={
                                    cells={anchor=east},
                                    legend pos=outer north east,
                            }]%,xtick=\empty,ytick=\empty,ztick=\empty ]
                            \addplot[color=astral, samples=500, domain=0:1, very thick] {x * sqrt(1 - x )};
                            \addlegendentry{$(\alpha, \beta)=(2,3/2)$}
                        \end{axis}
                    \end{tikzpicture}
                \end{center}
	}
\end{exo}


\begin{exo}{}
	Soit $X$ et $Y$ deux \vas indépendantes de fonction de répartition respective $F_X$ et $F_Y$. On pose $Z = \max(X,Y)$.
	\begin{enumerate}
		\item Déterminer la \fdr $F_Z$ de $Z$.

		\cor{La \fdr de $Z$ s'écrit
		\[
		F_Z(t) = \P(Z \leq t) = \P(X \leq t, Y \leq t)\,, \quad t \in \R\,.
		\]
		L'indépendance de $X$ et $Y$ donne alors
		\[
		F_Z(t) = \P(X \leq t) \P(Y \leq t) = F_X(t) F_Y(t)\,, \quad t \in \R\,.
		\]}

		\item En déduire une méthode pour simuler une \va de fonction de répartition
		\[
		F(t) = \min(t,1) (1-e^{-t}) \one_{]0, \infty[}(t)\,.
		\]

		\cor{Il suffit de reconnaître la \fdr de $\max(X,Y)$ où $X \sim \mathcal{U}([0,1])$ et $Y \sim \mathcal{E}(1)$ sont indépendantes. On part donc d'une \va $X$ de loi uniforme et d'une \va $Y$ de loi exponentielle (obtenue par exemple via la méthode d'inversion). On définit alors $Z = \max(X,Y)$ qui suit la loi donnée par $F$.}

		\item Déterminer la densité de cette \va.

		\cor{La densité $f$ de $Z$ correspond à la dérivée de la \fdr $F$ en tout point où celle-ci est dérivable. Ici, $F$ est dérivable pour tout $t \in \R \setminus \{0,1\}$. Si $t < 0$, alors $F'(t) = 0$. Si $t \in ]0,1[$, alors $F'(t) = 1 - e^{-t} + t e^{-t}$. Enfin, si $t > 1$, alors $F'(t) = e^{-t}$.
                \begin{center}
                    \begin{tikzpicture}[scale=1]
                        \begin{axis}[,xlabel=$t$,
                                ylabel={$f_Z(t)$},
                                width=13cm,
                                height=6cm,
                                ymin=0,
                                xmin=-5,
                                ymax=1.2,
                                xmax=5,
                                grid=major,
                                legend style={
                                    cells={anchor=east},
                                    legend pos=outer north east,
                            }]%,xtick=\empty,ytick=\empty,ztick=\empty ]
                            \addplot[color=astral, samples=5, domain=-5:0, very thick] {0};
                            \addplot[color=astral, samples=500, domain=0:1, very thick] {1 - exp(-x) + x * exp(-x)};
                            \addplot[color=astral, samples=500, domain=1:5, very thick] { exp(-x)};
                            %\addlegendentry{$(\alpha, \beta)=(2,3/2)$}
                        \end{axis}
                    \end{tikzpicture}
                \end{center}
		}
	\end{enumerate}
\end{exo}


\begin{exo}{}
	Soit $\lambda \in ]0,1[$ fixé.
	\begin{enumerate}
		\item On considère une \va discrète $Y$ de loi
		\[
		\P(Y=n) = (1-\lambda) \lambda^n\,, \quad n \in \N\,.
		\]
		Exprimer cette loi en fonction d'une loi connue.

		\cor{On pose $Z = Y+1$ et alors $Z$ est à valeurs dans $\N^*$ de loi
		\[
		\P(Z=n) = \P(Y=n-1) = (1-\lambda) \lambda^{n-1}\,, \quad n \geq 1\,.
		\]
		Ainsi, $Z$ suit une loi géométrique de paramètre $1-\lambda$ et $Y=Z-1$.}
		\item En se basant sur la méthode de rejet vue dans le cadre de loi à densité, proposer une méthode pour simuler une \va $X$ de loi de Poisson de paramètre $\lambda$ à partir de $Y$.

		\cor{Il suffit d'adapter ce qui a été vu en cours pour des \vas à densité, sauf qu'ici les densités sont calculées par rapport à la mesure de comptage sur $\N$. On remarque que pour tout entier $n$,
			\[
			p_n:= \P(X=n)
			= \dfrac{e^{-\lambda} \lambda^n}{n!}
			= \dfrac{e^{-\lambda}}{(1-\lambda) n!} \P(Y=n)
			=: \dfrac{e^{-\lambda}}{(1-\lambda) n!} q_n\,,
			\]
		Il suffit alors de choisir $m$ tel que $m \geq \frac{e^{-\lambda}}{(1-\lambda) n!}$, par exemple
		\[
		m = \dfrac{e^{-\lambda}}{1-\lambda}\,.
		\]
		On a alors bien $p_n \leq m q_n$ et le rapport d'acceptation est la suite $(r_n)$ donnée par
		\[
		r_n = \frac{p_n}{mq_n} = \frac{1}{n!}\,.
		\]
		La méthode de rejet consiste alors à simuler $U \sim \mathcal{U}([0,1])$ et $Y$ de loi donnée à la question 1, puis de tester si $U \leq r_Y = 1/(Y!)$.
		}
		\item Quelle est la probabilité de rejet ?

		\cor{On refait le calcul vu en cours. La probabilité de rejet vaut
		\begin{align*}
			\P(U > r_Y)
			&= \P((U,Y) \in \{(u,n) \in [0,1] \times \N: u > r_n\})\\
			&= \int_{\R \times \N} \one_{\{u > r_n\}} \, \mathrm d \P_{(U,Y)}(u,n)\\
			&= \sum_{n=0}^\infty \int_\R \one_{\{u > r_n\}} (1-\lambda) \lambda^n \one_{[0,1]}(u) \, \mathrm d u\,.
		\end{align*}
		On obtient alors
		\[
		\P(U > r_Y)
		= \sum_{n=0}^\infty  (1-\lambda) \lambda^n \int_{r_n}^1\, \mathrm d u
		= \sum_{n=0}^\infty  (1-\lambda) \lambda^n (1-r_n)\,.
		\]
		Par définition du rapport d'acceptation $r_n$, on en déduit que
		\[
		\P(U > r_Y)
		= \sum_{n=0}^\infty  p_n +  \sum_{n=0}^\infty \frac{q_n}{m}
		= 1 - \frac{1}{m}
		= 1 - (1-\lambda)e^\lambda\,.
		\]
		Plus $\lambda$ est proche de $0$, plus cette probabilité est petite.}
	\end{enumerate}
\end{exo}


\begin{exo}{}
	Vous disposez d'une pièce équilibrée. Proposez une méthode pour simuler le résultat d'un dé à 6 faces.

	\cor{On reprend le principe du rejet discret énoncé dans l'exercice précédent. On considère trois lancers consécutifs dont on note $\{0,\ldots,7\}$ les 8 résultats possibles (Pile-Pile-Pile, Face-Pile-Pile, Face-Face-Pile, etc.), par exemple sous forme binaire avec $0=$ Face et $1=$ Pile. On note $Y$ la \va prenant ces valeurs: elle suit une loi uniforme sur $\{0, \ldots, 7\}$. Le lancer d'un dé correspond à une loi uniforme sur $\{1, \ldots, 6\}$, on en déduit que si $X$ est la \va donnant le résultat du lancer d'un dé, on a l'inégalité
	\[
	\P(X = n) = \dfrac{1}{6} = \dfrac{8}{6} \times \P(Y = n)\,, \quad n =0,\ldots,7\,.
	\]
	On choisit donc $m=8/6$. La simulation s'effectue alors de la manière suivante: on lance trois fois notre pièce et on note le résultat obtenu (sous forme binaire). Si ce résultat est compris entre $1$ et $6$, on a obtenu une réalisation d'un lancer de dé, sinon on répète l'opération.

	Notons que la probabilité de rejet vaut $2/8 = 1/4$, ce qui est plutôt faible.
	}
\end{exo}



\begin{exo}{}
	On considère deux \vas $U$ et $V$ indépendantes de loi uniforme sur $[0,1]$.
	\begin{enumerate}
		\item Proposer une méthode pour simuler une \va $X$ de loi uniforme sur $[a,b]$, avec $a < b$.

		\cor{La \va $X = (b-a)U+a$ suit une loi uniforme sur $[a,b]$. En effet, la fonction de répartition de $X$ vaut
		\begin{align*}
		F_X(x)
		&= \P(X \leq x)
		= \P((b-a)U+a \leq x)
		= \P \Big(U \leq \frac{x-a}{b-a} \Big)\\
		&= \frac{x-a}{b-a} \one_{[0,1]}\Big(\frac{x-a}{b-a}\Big) + \one_{\{\frac{x-a}{b-a} > 1\}}
		= \frac{x-a}{b-a} \one_{[a,b]}(x) + \one_{\{x > b\}}\,.
		\end{align*}
		On reconnaît la fonction de répartition d'une loi uniforme sur $[a,b]$.}
		\item Proposer une méthode pour simuler un couple de \vas $(X,Y)$ de loi uniforme sur le pavé $[a,b]\times [c,d]$, avec $a < b$ et $c<d$.

		\cor{On pose $X = (b-a)U+a$ et $Y = (d-c)V+c$. La question précédente assure que $X \sim \mathcal{U}([a,b])$ et $Y \sim \mathcal{U}([c,d])$. Par ailleurs, les \vas $U$ et $V$ étant indépendantes, $X$ et $Y$ le sont également. Ainsi, $(X,Y)$ suit bien une loi uniforme sur la pavé $[a,b]\times [c,d]$.}

		\item Sans utiliser la méthode d'inversion, proposez une méthode pour simuler une \va de loi uniforme discrète sur $\{1, \ldots, n\}$.

		\cor{On pose $X = \lceil n U \rceil$. Comme $U$ suit une loi uniforme sur $[0,1]$, la \va $X$ prend les valeurs $1, \ldots, n$ avec, pour tout $k=1, \ldots, n$,
		\[
		\P(X=k)
		= \P( \lceil n U \rceil = k)
		= \P\Big( \frac{k-1}{n} < U \leq \frac{k}{n}\Big)
		= \frac{k}{n} - \frac{k-1}{n}
		= \frac{1}{n}\,.
		\]
		Ainsi, $X$ suit bien une loi uniforme sur $\{1, \ldots, n\}$.}

		\item Proposer une méthode de rejet pour simuler une \va uniforme sur le disque unité à partir de $U$ et $V$. Quelle est la probabilité de rejet ?

		\cor{À partir de $U$ et $V$ on obtient une \va uniforme sur le pavé $[-1,1]^2$ en considérant $X = 2U-1$ et $Y = 2V-1$. On considère les densités $f$ et $g$ des lois uniformes sur le disque $\mathcal{D}$ et sur le pavé $[-1,1]^2$:
		\[
		g(x,y) = \dfrac{\one_{\mathcal{D}}(x,y)}{\pi}
		\quad \text{et} \quad
		f(x,y) = \dfrac{\one_{[-1,1]^2}(x,y)}{4}\,.
		\]
		L'inégalité $g \leq \frac{4}{\pi} f$ incite à poser $m=\frac{4}{\pi}$ et à considérer le rapport d'acceptation $r(x,y) = \one_{\mathcal{D}}(x,y)$. La méthode de rejet revient alors à tester si une \va uniforme $A$ sur $[0,1]$ vérifie $A \leq\one_{\mathcal{D}}(X,Y)$: l'inégalité est vraie si $(X,Y)$ est dans $\mathcal{D}$ et fausse sinon (en fait $A$ ne sert à rien). Bref, il suffit de simuler $(X,Y)$ dans le pavé et de garder ce point s'il tombe dans le disque. La probabilité de rejet correspond alors à la probabilité de tomber hors du disque, soit $1-\pi/4 \approx 0.21$.
		}
		\item Montrer que le couple de \vas $(X,Y) = (\sqrt U \cos(2\pi V), \sqrt U \sin(2\pi V))$ suit également une loi uniforme sur le disque unité. On pourra utiliser la technique de la fonction muette.

		\cor{Soit $h: \R^2 \to \R$ une fonction mesurable bornée. Alors
		\[
		\E[h(X,Y)]
		= \int_{\R^2} h(\sqrt{u} \cos(2 \pi v), \sqrt{u} \sin(2 \pi v)) \one_{[0,1]}(u) \one_{[0,1]}(v)\, \mathrm du \mathrm dv\,.
		\]
		Le changement de variables $(r, \theta) = (\sqrt{u}, 2 \pi v) \iff (r^2, \frac{\theta}{2\pi}) = (u,v)$ donne alors
		\[
		\int_{[0,1]^2} h(\sqrt{u} \cos(2 \pi v), \sqrt{u} \sin(2 \pi v)) \, \mathrm du \mathrm dv
		= \int_0^1 \int_0^{2 \pi}  h(r \cos(\theta), r \sin(\theta)) \dfrac{r}{\pi}\, \mathrm dr \mathrm d\theta\,.
		\]
		Dans cette dernière expression, on reconnaît la transformation polaire vue en cours, ici restreinte au rayon $r \leq 1$. On obtient ainsi l'égalité
		\[
		\E[h(X,Y)]
		= \int_0^1 \int_0^{2 \pi}  h(r \cos(\theta), r \sin(\theta) \dfrac{r}{\pi}\, \mathrm dr \mathrm d\theta
		= \dfrac{1}{\pi} \int_{\mathcal{D}} h(x,y) \, \mathrm dx \mathrm dy\,.
		\]
		Ceci prouve que la densité de $(X,Y)$ est $ \frac{\one_{\mathcal{D}}(x,y)}{\pi}$, donc que ce vecteur suit une loi uniforme sur le disque.
		}
	\end{enumerate}
\end{exo}


\begin{exo}{}
	Soit $X$ une \va de densité $f: x \mapsto \max(0, 1-|x|)$. Construire une méthode de simulation de $X$ à l'aide de
	\begin{enumerate}
		\item la méthode d'inversion,
	\cor{ Le calcul de la fonction de répartition ne pose pas de problème:
	\[
	F_X(x)
	=
	\left\{ \begin{array}{llll}
		0 & \mbox{si } x < -1, \\
		\frac{1}{2}+ x + \frac{x^2}{2} & \mbox{si } -1 \leq x < 0,\\
		\frac{1}{2} + x - \frac{x^2}{2} & \mbox{si } 0 \leq x \leq 1,\\
		1 & \mbox{si } x > 1.
	\end{array}
	\right.
	\]

	Si $u \in ]0,1/2[$, alors l'équation $F_X(x)=u$ se réécrit
	\[
	x^2 + 2x +1-2u = 0\,.
	\]
	Le discriminant de ce trinôme est $8u$ et l'unique solution dans $]-1,0[$ est alors $F_X^\leftarrow(u) = \sqrt{2u} -1$. De manière analogue, on trouve pour le cas $u \in ]1/2,1[$ la solution $F_X^\leftarrow(u) = 1-\sqrt{2(1-u)}$.

    La méthode d'inversion consiste alors à simuler une \va uniforme $U$ sur $[0,1]$ et à poser $X = F_X^\leftarrow(U)$.}
		\item la méthode de rejet.

    \cor{

	 Pour la méthode de rejet, on choisit une densité uniforme $g$ sur $[-1,1]$ qui vérifie
	\[
	f(x) = \max(0, 1-|x|) \leq 2 \dfrac{\one_{[-1,1]}(x)}{2} = 2 g(x)\,, \quad x \in \R\,.
	\]
	On peut donc choisir ici $m=2$. La méthode de rejet consiste alors à simuler une loi uniforme $U$ sur $[0,1]$ et une loi uniforme $Y$ sur $[-1,1]$, et à tester si $U \leq 1-|Y|$. Si oui on garde $Y$, sinon on répète l'opération.}
	\end{enumerate}
\end{exo}


\begin{exo}{}
	Soit $X$ un vecteur aléatoire sur $\R^d$ et $B \in \mathcal{B}(\R^d)$ un borélien de $\R^d$. On suppose que $p = \P(X \in B) > 0$. On rappelle que la loi de $X$ sachant $B$ est définie via
	\[
	\P_{X \mid B}(A) = \P(X \in A \mid X \in B) = \dfrac{\P(X \in A \cap B)}{p}\,, \quad A \in \mathcal{B}(\R^d)\,.
	\]
	On suppose que l'on sait générer suivant la loi de $X$ et on souhaite générer suivant la loi $\P_{X \mid B}$.

	\begin{enumerate}
		\item Soit $(X_n)_{n \geq 1}$ des \vas suivant la loi de $X$. Déterminer la loi de
		\[
		N = \min\{n \geq 1: X_n \in B\}\,.
		\]

		\cor{Soit $n \geq 1$. L'\ev $\{N=n\}$ est réalisé si et seulement si les $n-1$ premières \vas $X_i$ ne sont pas dans $B$ et que $X_n$ l'est. Ainsi, pour tout $n \geq 1$,
		\[
		\P(N=n)
		= \P(X_1 \notin B, \ldots, X_{n-1} \notin B, X_n \in B)
		\]
		Par indépendance des $X_k$, on obtient alors
		\[
		\P(N=n)
		= \P(X_1 \notin B) \cdots \P(X_{n-1} \notin B) \P(X_n \in B)
		= (1-p)^{n-1} p\,,
		\]
		ce qui prouve que $N$ suit une loi géométrique de paramètre $p$.}
		\item Montrer que $X_N$ suit la loi $\P_{X \mid B}$. En déduire une méthode pour simuler une \va de loi $\P_{X \mid B}$.

		\cor{En reprenant mutatis mutandis le calcul précédent, on obtient la relation
		\[
		\P(N=n, X_n \in A)
		=  \P(X_1 \notin B) \cdots \P(X_{n-1} \notin B) \P(X_n \in A \cap B)
		= (1-p)^{n-1} \P(X \in A \cap B)\,.
		\]
		On conclut alors par $\sigma$-additivité:
		\[
		\P(X_N \in A)
		= \sum_{n=1}^\infty \P(X_N \in A, N = n)
		=\sum_{n=1}^\infty (1-p)^{n-1} \P(X \in A \cap B)\,.
		\]
		On reconnaît la somme d'une série géométrique:
		\[
		\P(X_N \in A)
		= \dfrac{\P(X \in A \cap B)}{p}
		= \P_B(X \in A)\,.
		\]
		En pratique, on simule $X$ et si $X \in B$ on le garde, sinon on simule une nouvelle réalisation de $X$ et on répète la procédure.}
	\end{enumerate}
\end{exo}

\begin{exo}{}
    Soit $F$ une fonction de répartition continue.
	\begin{enumerate}
		\item Montrer l'égalité $F\circ F^\leftarrow (u) = u$ pour tout $u \in ]0,1]$.

		\cor{Rappelons l'équivalence prouvée en cours. Pour $u \in ]0,1[$ et $x \in \R$, on a
		\[
		F^\leftarrow(u) \leq x \iff u \leq F(x)\,.
		\]
		En prenant $x=F^\leftarrow(u)$ on obtient alors $u \leq F(F^\leftarrow(u))$. De même, en prenant $x=F^\leftarrow(u) - \epsilon$, pour $\epsilon > 0$, on obtient $u > F(F^\leftarrow(u)-\epsilon)$. Comme $F$ est supposée continue en $F^\leftarrow(u) > 0$, on obtient, en faisant tendre $\epsilon$ vers $0$, l'inégalité $u \geq F(F^\leftarrow(u))$. On conclut alors que $F(F^\leftarrow (u)) = u$.

		Si $u=1$, alors soit $F^\leftarrow(u) = \infty$ et l'égalité $F(F^\leftarrow (u)) = u$ est bien vérifiée, soit $F^\leftarrow(u) < \infty$ et on reprend la démonstration précédente.
		}

		\item Soit $X$ une \va ayant une \fdr $F_X$ continue. Montrer que la \va $F_X(X)$ suit une loi uniforme sur $[0,1]$.

		\cor{On considère une \va $U$ uniforme sur $[0,1]$. On sait d'après le cours que $X \stackrel{\mathcal{L}}{=} F_X^\leftarrow(U)$. La question précédente assure alors que $F_X(X) \stackrel{\mathcal{L}}{=} F_X(F_X^\leftarrow(U)) = U$, ce qui prouve que $F_X(X)$ suit une loi uniforme sur $[0,1]$.}

		\item Que se passe-t-il si on enlève l'hypothèse de continuité de $F_X$ ?

		\cor{Supposons que $F_X$ présente une discontinuité en $x_0$. Alors $X$ ne prend aucune valeur comprise entre $F_X(x_0-)$ et $F(x_0)$, donc la variable $F_X(X)$ non plus: elle ne peut alors pas suivre une loi uniforme. Par exemple, si $X$ suit une loi de Bernoulli de paramètre $1/2$, alors $F_X(X)$ ne prend que les valeurs $1/2$ et $1$, et ce de manière équiprobable.}
	\end{enumerate}
\end{exo}

\end{document}
